%===============================================================================
% LaTeX sjabloon voor de bachelorproef toegepaste informatica aan HOGENT
% Meer info op https://github.com/HoGentTIN/latex-hogent-report
%===============================================================================

\documentclass[dutch,dit,thesis]{hogentreport}

% TODO:
% - If necessary, replace the option `dit`' with your own department!
%   Valid entries are dbo, dbt, dgz, dit, dlo, dog, dsa, soa
% - If you write your thesis in English (remark: only possible after getting
%   explicit approval!), remove the option "dutch," or replace with "english".

\usepackage{lipsum} % For blind text, can be removed after adding actual content

%% Pictures to include in the text can be put in the graphics/ folder
\graphicspath{{graphics/}}

%% For source code highlighting, requires pygments to be installed
%% Compile with the -shell-escape flag!
\usepackage[section]{minted}
%% If you compile with the make_thesis.{bat,sh} script, use the following
%% import instead:
%% \usepackage[section,outputdir=../output]{minted}
\usemintedstyle{solarized-light}
\definecolor{bg}{RGB}{253,246,227} %% Set the background color of the codeframe

%% Change this line to edit the line numbering style:
\renewcommand{\theFancyVerbLine}{\ttfamily\scriptsize\arabic{FancyVerbLine}}

%% Macro definition to load external java source files with \javacode{filename}:
\newmintedfile[javacode]{java}{
    bgcolor=bg,
    fontfamily=tt,
    linenos=true,
    numberblanklines=true,
    numbersep=5pt,
    gobble=0,
    framesep=2mm,
    funcnamehighlighting=true,
    tabsize=4,
    obeytabs=false,
    breaklines=true,
    mathescape=false
    samepage=false,
    showspaces=false,
    showtabs =false,
    texcl=false,
}

% Other packages not already included can be imported here

%%---------- Document metadata -------------------------------------------------
% TODO: Replace this with your own information
\author{Jona Fouquaert}
\supervisor{Dhr. S. De Gheselle}
\cosupervisor{Dhr. C. Stal}
\title{Diepgaande Analyse van Geospatiale Veranderingen: Een Vergelijkende Studie van Machine Learning Modellen in Remote Sensing Change Detection voor de Vegetatie van Gent (2020-2024)}
\academicyear{\advance\year by -1 \the\year--\advance\year by 1 \the\year}
\examperiod{1}
\degreesought{\IfLanguageName{dutch}{Professionele bachelor in de toegepaste informatica}{Bachelor of applied computer science}}
\partialthesis{false} %% To display 'in partial fulfilment'
%\institution{Internshipcompany BVBA.}

%% Add global exceptions to the hyphenation here
\hyphenation{back-slash}

%% The bibliography (style and settings are  found in hogentthesis.cls)
\addbibresource{bachproef.bib}            %% Bibliography file
\addbibresource{../voorstel/voorstel.bib} %% Bibliography research proposal
\defbibheading{bibempty}{}

%% Prevent empty pages for right-handed chapter starts in twoside mode
\renewcommand{\cleardoublepage}{\clearpage}

\renewcommand{\arraystretch}{1.2}

%% Content starts here.
\begin{document}

%---------- Front matter -------------------------------------------------------

\frontmatter

\hypersetup{pageanchor=false} %% Disable page numbering references
%% Render a Dutch outer title page if the main language is English
\IfLanguageName{english}{%
    %% If necessary, information can be changed here
    \degreesought{Professionele Bachelor toegepaste informatica}%
    \begin{otherlanguage}{dutch}%
       \maketitle%
    \end{otherlanguage}%
}{}

%% Generates title page content
\maketitle
\hypersetup{pageanchor=true}

%%=============================================================================
%% Voorwoord
%%=============================================================================

\chapter*{\IfLanguageName{dutch}{Woord vooraf}{Preface}}%
\label{ch:voorwoord}

%% TODO:
%% Het voorwoord is het enige deel van de bachelorproef waar je vanuit je
%% eigen standpunt (``ik-vorm'') mag schrijven. Je kan hier bv. motiveren
%% waarom jij het onderwerp wil bespreken.
%% Vergeet ook niet te bedanken wie je geholpen/gesteund/... heeft

\lipsum[1-2]
%%=============================================================================
%% Samenvatting
%%=============================================================================

% TODO: De "abstract" of samenvatting is een kernachtige (~ 1 blz. voor een
% thesis) synthese van het document.
%
% Een goede abstract biedt een kernachtig antwoord op volgende vragen:
%
% 1. Waarover gaat de bachelorproef?
% 2. Waarom heb je er over geschreven?
% 3. Hoe heb je het onderzoek uitgevoerd?
% 4. Wat waren de resultaten? Wat blijkt uit je onderzoek?
% 5. Wat betekenen je resultaten? Wat is de relevantie voor het werkveld?
%
% Daarom bestaat een abstract uit volgende componenten:
%
% - inleiding + kaderen thema
% - probleemstelling
% - (centrale) onderzoeksvraag
% - onderzoeksdoelstelling
% - methodologie
% - resultaten (beperk tot de belangrijkste, relevant voor de onderzoeksvraag)
% - conclusies, aanbevelingen, beperkingen
%
% LET OP! Een samenvatting is GEEN voorwoord!

%%---------- Nederlandse samenvatting -----------------------------------------
%
% TODO: Als je je bachelorproef in het Engels schrijft, moet je eerst een
% Nederlandse samenvatting invoegen. Haal daarvoor onderstaande code uit
% commentaar.
% Wie zijn bachelorproef in het Nederlands schrijft, kan dit negeren, de inhoud
% wordt niet in het document ingevoegd.

% \IfLanguageName{english}{%
% \selectlanguage{dutch}
% \chapter*{Samenvatting}
% \lipsum[1-4]
% \selectlanguage{english}
% }{}

%%---------- Samenvatting -----------------------------------------------------
% De samenvatting in de hoofdtaal van het document

\chapter*{\IfLanguageName{dutch}{Samenvatting}{Abstract}}

Het abstract wordt uitgewerkt in een latere versie van de BP

% Geospatiale remote sensing biedt innovatieve methoden voor het monitoren van veranderingen in het landschap. 
% Dit onderzoek richt zich specifiek op de detectie en visualisatie van veranderingen in vegetatiebedekking in de stad Gent tussen 2020 en 2024.
% Hiervoor worden Sentinel-2 satellietbeelden gebruikt en geanalyseerd met behulp van machine learning-technieken. 
% Een geavanceerde machine learning-pijplijn wordt ontwikkeld om deze veranderingen te detecteren en te interpreteren.
% De onderzoeksmethode omvat een vergelijkende studie van diverse algoritmen, waaronder Support Vector Machines (SVM), Random Forests, 
% Convolutional Neural Networks (CNN), Siamese Networks en Vision Transformers. Preprocessing van de data, 
% zoals herschalen en semantische segmentatie, zorgt ervoor dat de inputdata consistent en geschikt is voor training. 
% De algoritmen worden beoordeeld op nauwkeurigheid, consistentie en vermogen om visueel interpreteerbare resultaten te genereren.
% Een belangrijke evaluatie metriek is de Vegetation Condition Index (VCI), die veranderingen in vegetatiebedekking kwantificeert door 
% Normalized Difference Vegetation Index (NDVI)-waarden te vergelijken. Daarnaast worden metrieken zoals precision, recall en 
% Intersection over Union (IoU) toegepast om de prestaties van de modellen te beoordelen. Het onderzoek beoogt niet alleen inzicht te bieden 
% in de geschiktheid van verschillende machine learning-modellen voor geospatiale change detection, maar ook bij te dragen aan de 
% toepassing van AI in stedelijke milieuanalyse en besluitvorming. Het onderzoek fungeert ook als oproep tot actie voor natuurbehoud in steden.


%---------- Inhoud, lijst figuren, ... -----------------------------------------

\tableofcontents

% In a list of figures, the complete caption will be included. To prevent this,
% ALWAYS add a short description in the caption!
%
%  \caption[short description]{elaborate description}
%
% If you do, only the short description will be used in the list of figures

\listoffigures

% If you included tables and/or source code listings, uncomment the appropriate
% lines.
%\listoftables
%\listoflistings

% Als je een lijst van afkortingen of termen wil toevoegen, dan hoort die
% hier thuis. Gebruik bijvoorbeeld de ``glossaries'' package.
% https://www.overleaf.com/learn/latex/Glossaries

%---------- Kern ---------------------------------------------------------------

\mainmatter{}

% De eerste hoofdstukken van een bachelorproef zijn meestal een inleiding op
% het onderwerp, literatuurstudie en verantwoording methodologie.
% Aarzel niet om een meer beschrijvende titel aan deze hoofdstukken te geven of
% om bijvoorbeeld de inleiding en/of stand van zaken over meerdere hoofdstukken
% te verspreiden!

%%=============================================================================
%% Inleiding
%%=============================================================================

\chapter{\IfLanguageName{dutch}{Inleiding}{Introduction}}%
\label{ch:inleiding}

% De inleiding moet de lezer net genoeg informatie verschaffen om het onderwerp te begrijpen en in te zien waarom de onderzoeksvraag de moeite waard is om te onderzoeken. In de inleiding ga je literatuurverwijzingen beperken, zodat de tekst vlot leesbaar blijft. Je kan de inleiding verder onderverdelen in secties als dit de tekst verduidelijkt. Zaken die aan bod kunnen komen in de inleiding~\autocite{Pollefliet2011}:

% \begin{itemize}
%   \item context, achtergrond
%   \item afbakenen van het onderwerp
%   \item verantwoording van het onderwerp, methodologie
%   \item probleemstelling
%   \item onderzoeksdoelstelling
%   \item onderzoeksvraag
%   \item \ldots
% \end{itemize}

% Geo-ICT is de toepassing van informatie- en communicatietechnologie in de geografie. Een belangrijk component van Geo-ICT is het programma GIS.
% Geografische informatiesystemen (GIS) zijn digitale systemen die geografische gegevens analyseren, visualiseren en beheren om ruimtelijke 
% patronen en relaties te begrijpen. Het wordt vandaag de dag door bedrijven gebruikt in verschillende sectoren. 
% Nu er vraag is naar GIS-applicaties, zijn er ook ondernemingen die zich specialiseren in deze trend, zoals bijvoorbeeld het bedrijf GeoAI. 
% Zij combineren Geo-ICT met het experimentele veld van machine learning. Dit kan ingezet worden om verschillende doeleinden te bereiken, zoals
% bijvoorbeeld automatische kaartgeneratie, voorspellende modellering, enz\ldots.
% \newline
% Het thema waar ik me op focus is geospatiale remote sensing change detection. Dit richt zich op het ontdekken van veranderingen in een bepaald landschap.
% Hier komt de sterkte van AI naar boven. Het is in staat patronen te ontdekken die wij als mensen moeilijk begrijpen.
% Deze studie is gericht op het bestuderen van de aanwezigheid van vegetatie in de stad Gent tussen de periode van 2020-2024. 
% De dataset bestaat uit luchtfoto's genomen door het agentschap digitaal Vlaanderen (middenschalige orthofotobedekking van het Vlaamse Gewest). 
% Tijdens het onderzoek worden verschillende machine learning algoritmes (Support Vector Machines (SVM), Random Forests, 
% Convolutional Neural Networks (CNNs), Siamese Networks en Vision transformer) getest op de dataset. 
% Het model moet in staat zijn vegetatie te detecteren en ook het verschil tussen de twee periodes te visualiseren. 
% Het open-sourceplatform QGIS wordt gebruikt om de data te verkrijgen en te visualiseren.

% Change detection (CD) is het waarnemen van verandering in een bepaald gebied aan de hand van afbeeldingen genomen op verschillende tijden. 
% Machine learning wordt al geruime tijd toegepast op CD. Vroeger lag de focus meer op tradionele pixel based methodes. 
% Maar in de laatste jaren heeft de opkomst van verschillende nieuwere algoritmen het veld verbreed. Zo worden verschillende studies gedaan 
% rond de impact van deep learning neurale netwerken op CD, zoals het onderzoek van Bai e.a. (2022) over Deep Learning Change Detection (DLCD).

\section{\IfLanguageName{dutch}{Probleemstelling}{Problem Statement}}%
\label{sec:probleemstelling}

% Uit je probleemstelling moet duidelijk zijn dat je onderzoek een meerwaarde heeft voor een concrete doelgroep. De doelgroep moet goed gedefinieerd en afgelijnd zijn. Doelgroepen als ``bedrijven,'' ``KMO's'', systeembeheerders, enz.~zijn nog te vaag. Als je een lijstje kan maken van de personen/organisaties die een meerwaarde zullen vinden in deze bachelorproef (dit is eigenlijk je steekproefkader), dan is dat een indicatie dat de doelgroep goed gedefinieerd is. Dit kan een enkel bedrijf zijn of zelfs één persoon (je co-promotor/opdrachtgever).

Klimaatverandering is onbetwistbaar het centrale thema van de 21e eeuw, met de opwarming van de aarde die een scala aan negatieve 
effecten met zich meebrengt, waaronder stijgende zeespiegels en drogere klimaten. Deze droogte draagt bij aan een toename van 
bosbranden wereldwijd, wat des te zorgwekkender is gezien de cruciale rol die bossen spelen bij het tegengaan van klimaatverandering. 
Tegelijkertijd worden we geconfronteerd met een groeiende overbevolking, wat resulteert in een toenemende behoefte aan huisvesting 
en de daaropvolgende afname van natuurlijke habitat ‘s wereldwijd. Het is daarom van essentieel belang om stedelijke ontwikkeling en behoud 
van natuurlijke omgevingen hand in hand te laten gaan.
\newline
\newline
In 2022 stelde de Europese Commissie de natuurherstel wet voor. Die verplicht lidstaten om tegen 2030 tot 30 procent van 
verwaarloosde ecosystemen te herstellen. De wet werd uiteindelijk goedgekeurd op 17 juni 2024. Een bijkomende maatregel in dit besluit is een
oproep naar meer groene ruimtes in steden.
Verschillende onderzoeken zoals \textcite{bajirao2015importance} en \textcite{Birch_2020} tonen aan dat een groene omgeving een positief effect heeft op 
menselijk welzijn. Tijdens de zomermaanden zorgt stedelijke infrastructuur voor 
hogere temperaturen. In contrast blijkt dat vegetatie juist tot een lagere omgevingstemperatuur leidt \autocite{vuckovic2017studies}.
In België en meer specifiek Vlaanderen is er ook een groot grondwater probleem. Door de verharding van de grond sijpelt regenwater moeilijker door. 
Dit heeft negatieve gevolgen zoals mogelijke overstromingen. Meer vegetatie zorgt voor betere grondwater doorsijpeling. 
Vegetatie bevordert ook de zuivering van de lucht. Luchtvervuiling is ondanks lage-emissiezones in steden, nog altijd een probleem.
Meer vegetatie betekend een betere luchtkwaliteit.
\newline
\newline
Deze bevindingen onderstrepen het belang van het integreren van groene ruimtes in stadsplanning en ontwikkeling. 
In lijn hiermee richt deze studie zich op het observeren van de ontwikkeling van vegetatie in stedelijke omgevingen, met als doel inzicht 
te krijgen in hoe deze omgevingen evolueren en hoe ze kunnen worden verbeterd ten behoeve van zowel de menselijke gezondheid als de natuur.
Dit onderzoek is gericht naar steden en gemeentes om een beter inzicht te krijgen wat voor impact beleid heeft op de natuur.
\newline
\newline

% Klimaatverandering is onbetwistbaar het centrale thema van de 21e eeuw, met de opwarming van de aarde die een scala aan negatieve 
% effecten met zich meebrengt, waaronder stijgende zeespiegels en drogere klimaten. Deze droogte draagt bij aan een toename van 
% bosbranden wereldwijd, wat des te zorgwekkender is gezien de cruciale rol die bossen spelen bij het tegengaan van klimaatverandering. 
% Tegelijkertijd worden we geconfronteerd met een groeiende overpopulatie, wat resulteert in een toenemende behoefte aan huisvesting 
% en de daaropvolgende afname van natuurlijke habitats wereldwijd. Het is daarom van essentieel belang om stedelijke ontwikkeling en behoud 
% van natuurlijke omgevingen hand in hand te laten gaan.
% \newline
% \newline
% Bepaalde onderzoeken zoals \textcite{Birch_2020}, benadrukt de positieve invloed van natuurlijke omgevingen in stedelijke gebieden op 
% de mentale gezondheid van de bewoners. In België, en met name in Vlaanderen, speelt bovendien een ernstig probleem met betrekking tot grondwaterbeheer. 
% Door toenemende bodemverharding wordt de infiltratie van regenwater belemmerd, wat kan leiden tot negatieve gevolgen zoals een 
% verhoogd risico op overstromingen. Een toename van vegetatie bevordert echter de waterdoorlaatbaarheid van de bodem, waardoor deze 
% problematiek gedeeltelijk kan worden verzacht. Tegelijkertijd kan de aanwezigheid van stedelijke infrastructuur leiden tot verhoogde 
% temperaturen tijdens de zomermaanden. Daartegenover staat dat vegetatie doorgaans geassocieerd wordt met lagere omgevingstemperaturen 
% in vergelijking met stedelijke structuren \autocite{vuckovic2017studies}.
% \newline
% \newline
% Deze bevindingen onderstrepen het belang van het integreren van groene ruimtes in stadsplanning en ontwikkeling. 
% In lijn hiermee richt dit onderzoek zich op het observeren van de ontwikkeling van vegetatie in stedelijke omgevingen, met als doel inzicht 
% te krijgen in hoe deze omgevingen evolueren en hoe ze kunnen worden verbeterd ten behoeve van zowel de menselijke gezondheid als de natuur.

\section{\IfLanguageName{dutch}{Onderzoeksvraag}{Research question}}%
\label{sec:onderzoeksvraag}

% Wees zo concreet mogelijk bij het formuleren van je onderzoeksvraag. Een onderzoeksvraag is trouwens iets waar nog niemand op dit moment een antwoord heeft (voor zover je kan nagaan). Het opzoeken van bestaande informatie (bv. ``welke tools bestaan er voor deze toepassing?'') is dus geen onderzoeksvraag. Je kan de onderzoeksvraag verder specifiëren in deelvragen. Bv.~als je onderzoek gaat over performantiemetingen, dan 

% Deze studie is een verglijkende proef tussen verschillende machine learning modellen om veranderingen te detecteren aan vegetatie 
% in een stedelijke omgeving over een bepaalde tijdsperiode. De modellen zijn opgedeeld in twee groepen. De nieuwere deep learning modellen en de 
% origenele algoritmen.

Dit onderzoek richt zich op de vraag welk machine learning-model het meest geschikt is voor het nauwkeurig detecteren en visualiseren 
van veranderingen in de aanwezigheid van vegetatie in de stad Gent tussen 2020 en 2024, met behulp van remote sensing change detection.
\newline
Ondersteunend met deze hoofdvraag zoek ik naar antwoorden voor een aantal deelvragen. Welke objectklasse (vegetatie, bebouwing, wegen enz.) 
tonen de grootste veranderingen in percentage tussen de twee tijdstippen? Wat is het verschil tussen handmatige (traditionele ML) en 
automatische feature extractie (Deep Learning)? Wat voor preprocessing stappen zijn nodig voor een ideale dataset? Wat zijn de belangrijkste 
evaluatie criteria bij het beoordelen van een model? Is één dataset genoeg om tot een generaliseerbaar model te komen?

% Dit onderzoek richt zich op de vraag welk machine learning-model het meest geschikt is voor het nauwkeurig detecteren en visualiseren 
% van veranderingen in de aanwezigheid van vegetatie in de stad Gent tussen 2020 en 2024, met behulp van remote sensing change detection.
% Om deze hoofdvraag te beantwoorden, worden verschillende deelaspecten geanalyseerd. Allereerst wordt onderzocht welke objectklasse 
% (vegetatie, bebouwing of wegen) de grootste veranderingen vertoont in percentage tussen de twee onderzochte tijdstippen. 
% Daarnaast wordt nagegaan of er specifieke patronen in de vegetatieverandering te identificeren zijn, zoals groei, afname of stabiliteit 
% in bepaalde delen van de stad. Hierbij wordt tevens bekeken of er ruimtelijke structuren of clusters in deze veranderingen aanwezig zijn.
% \newline
% \newline
% Een ander belangrijk aspect van de studie is de vergelijking tussen handmatige feature-extractie en automatische extractie op basis van 
% deep learning, waarbij de impact van beide methoden op de uiteindelijke resultaten wordt geanalyseerd. Daarnaast wordt onderzocht welke 
% evaluatiemethoden het meest geschikt zijn om de prestaties van de modellen te beoordelen en hoe de balans tussen nauwkeurigheid en 
% andere prestatie-indicatoren, zoals recall en precision, wordt gemeten.
% \newline
% Verder wordt bekeken welke criteria bepalen of een model als `acceptabel' wordt beschouwd binnen de context van change detection en of 
% één dataset voldoende is om de generaliseerbaarheid van het model te waarborgen. Hierbij wordt ook aandacht besteed aan mogelijke bias 
% of beperkingen in de gebruikte datasets en hoe hiermee kan worden omgegaan. Tot slot wordt geanalyseerd welke preprocessing-stappen 
% nodig zijn om de datasets geschikt te maken voor input in de machine learning-modellen.
% \newline
% Door middel van deze deelvragen wordt een systematische benadering gehanteerd om de effectiviteit en toepasbaarheid van verschillende 
% machine learning-modellen voor geospatiale change detection te evalueren.



% De onderzoeksvraag: Welk machine learning-model is het meest geschikt voor het nauwkeurig detecteren en visualiseren van de
% veranderingen in aanwezigheid van vegetatie in de stad Gent tussen 2020 en 2024, met behulp van remote sensing change detection?
% \newline
% De deelvragen: 
% \newline
% - Welke object klasse (vegetatie, bebouwing, wegen enz \ldots) toont de grootste veranderingen in percentage tussen de twee tijdstippen?
% \newline
% - Zijn er specifieke patronen in de veranderingen, zoals groei, afname of stabiliteit van groen in bepaalde delen van de stad?
% \newline
% - Zijn er ruimtelijke patronen in de veranderingen van vegetatie in de stad Gent?
% \newline
% - Is er een verschil tussen handmatige feature-extractie en automatische extractie op basis van deep learning, en wat is de impact hiervan op de resultaten?
% \newline
% - Welke evaluatie methoden zijn het meest geschikt om de prestaties van het model te beoordelen?
% \newline
% - Hoe wordt de balans gemeten tussen nauwkeurigheid en andere prestatie-indicatoren, zoals recall of precision?
% \newline
% - Welke criteria bepalen of een model als `acceptabel' wordt beschouwd?
% \newline
% - Is één dataset voldoende om de generaliseerbaarheid van het model te garanderen?
% \newline
% - Hoe wordt omgegaan met mogelijke bias of beperkingen in de gebruikte datasets?
% \newline
% - Welke preprocessing-stappen zijn nodig om de datasets geschikt te maken voor model input?


\section{\IfLanguageName{dutch}{Onderzoeksdoelstelling}{Research objective}}%
\label{sec:onderzoeksdoelstelling}

% Wat is het beoogde resultaat van je bachelorproef? Wat zijn de criteria voor succes? Beschrijf die zo concreet mogelijk. Gaat het bv.\ om een proof-of-concept, een prototype, een verslag met aanbevelingen, een vergelijkende studie, enz.

% Het doel van deze studie is het beste model te selecteren voor een change detection architectuur om veranderingen in vegetatie te 
% detecteren in een stedelijke omgeving. De succescriteria is de accuraatheid van het model. De snelheid is ook belangrijk maar niet 
% doorslagevend.

Dit is een vergelijkende studie met als doel het beste model te selecteren voor een change detecion architectuur om veranderingen in
vegetatie te detecteren in een stedelijke omgeving. Het resultaat zal een machine learning pipeline zijn bestaande uit het best 
scorende model. 
\newline
% Voor de evaluatie van de verschillende machine learning-modellen worden meerdere prestatie-indicatoren gehanteerd. 
% De Vegetation Condition Index (VCI) wordt als primaire metriek gebruikt om veranderingen in vegetatiebedekking te kwantificeren door 
% de Normalized Difference Vegetation Index (NDVI)-waarden van t₁ en t₂ met elkaar te vergelijken. Daarnaast worden aanvullende 
% evaluatiemethoden toegepast, waaronder confusion matrix-gebaseerde metrieken zoals precision en recall, evenals Intersection over Union (IoU)
% om de nauwkeurigheid van de voorspelde veranderingen te beoordelen.

\section{\IfLanguageName{dutch}{Opzet van deze bachelorproef}{Structure of this bachelor thesis}}%
\label{sec:opzet-bachelorproef}

% Het is gebruikelijk aan het einde van de inleiding een overzicht te
% geven van de opbouw van de rest van de tekst. Deze sectie bevat al een aanzet
% die je kan aanvullen/aanpassen in functie van je eigen tekst.

De rest van deze bachelorproef is als volgt opgebouwd:
\newline
\newline
In Hoofdstuk~\ref{ch:stand-van-zaken} wordt een overzicht gegeven van de stand van zaken binnen het onderzoeksdomein, op basis van een literatuurstudie.

In Hoofdstuk~\ref{ch:methodologie} wordt de methodologie toegelicht en worden de gebruikte onderzoekstechnieken besproken om een antwoord te kunnen formuleren op de onderzoeksvragen.

In Hoofdstuk~\ref{ch:model-evaluatie} worden de verschillende evaluatie technieken besproken om de verandering tussen beide tijdstippen te bestuderen. 

In Hoofdstuk~\ref{ch:model-resultaten} worden de resultaten van alle modellen weergeven.

% TODO: Vul hier aan voor je eigen hoofstukken, één of twee zinnen per hoofdstuk

In Hoofdstuk~\ref{ch:conclusie}, tenslotte, wordt de conclusie gegeven en een antwoord geformuleerd op de onderzoeksvragen. Daarbij wordt ook een aanzet gegeven voor toekomstig onderzoek binnen dit domein.
\chapter{\IfLanguageName{dutch}{Stand van zaken}{State of the art}}%
\label{ch:stand-van-zaken}

% Tip: Begin elk hoofdstuk met een paragraaf inleiding die beschrijft hoe
% dit hoofdstuk past binnen het geheel van de bachelorproef. Geef in het
% bijzonder aan wat de link is met het vorige en volgende hoofdstuk.

% Pas na deze inleidende paragraaf komt de eerste sectiehoofding.

Dit hoofdstuk bevat je literatuurstudie. De inhoud gaat verder op de inleiding, maar zal het onderwerp van de bachelorproef *diepgaand* uitspitten. De bedoeling is dat de lezer na lezing van dit hoofdstuk helemaal op de hoogte is van de huidige stand van zaken (state-of-the-art) in het onderzoeksdomein. Iemand die niet vertrouwd is met het onderwerp, weet nu voldoende om de rest van het verhaal te kunnen volgen, zonder dat die er nog andere informatie moet over opzoeken \autocite{Pollefliet2011}.

Je verwijst bij elke bewering die je doet, vakterm die je introduceert, enz.\ naar je bronnen. In \LaTeX{} kan dat met het commando \texttt{$\backslash${textcite\{\}}} of \texttt{$\backslash${autocite\{\}}}. Als argument van het commando geef je de ``sleutel'' van een ``record'' in een bibliografische databank in het Bib\LaTeX{}-formaat (een tekstbestand). Als je expliciet naar de auteur verwijst in de zin (narratieve referentie), gebruik je \texttt{$\backslash${}textcite\{\}}. Soms is de auteursnaam niet expliciet een onderdeel van de zin, dan gebruik je \texttt{$\backslash${}autocite\{\}} (referentie tussen haakjes). Dit gebruik je bv.~bij een citaat, of om in het bijschrift van een overgenomen afbeelding, broncode, tabel, enz. te verwijzen naar de bron. In de volgende paragraaf een voorbeeld van elk.

\textcite{Knuth1998} schreef een van de standaardwerken over sorteer- en zoekalgoritmen. Experten zijn het erover eens dat cloud computing een interessante opportuniteit vormen, zowel voor gebruikers als voor dienstverleners op vlak van informatietechnologie~\autocite{Creeger2009}.

Let er ook op: het \texttt{cite}-commando voor de punt, dus binnen de zin. Je verwijst meteen naar een bron in de eerste zin die erop gebaseerd is, dus niet pas op het einde van een paragraaf.

\lipsum[7-20]

%%=============================================================================
%% Methodologie
%%=============================================================================

\chapter{\IfLanguageName{dutch}{Methodologie}{Methodology}}%
\label{ch:methodologie}

%% TODO: In dit hoofstuk geef je een korte toelichting over hoe je te werk bent
%% gegaan. Verdeel je onderzoek in grote fasen, en licht in elke fase toe wat
%% de doelstelling was, welke deliverables daar uit gekomen zijn, en welke
%% onderzoeksmethoden je daarbij toegepast hebt. Verantwoord waarom je
%% op deze manier te werk gegaan bent.
%% 
%% Voorbeelden van zulke fasen zijn: literatuurstudie, opstellen van een
%% requirements-analyse, opstellen long-list (bij vergelijkende studie),
%% selectie van geschikte tools (bij vergelijkende studie, "short-list"),
%% opzetten testopstelling/PoC, uitvoeren testen en verzamelen
%% van resultaten, analyse van resultaten, ...
%%
%% !!!!! LET OP !!!!!
%%
%% Het is uitdrukkelijk NIET de bedoeling dat je het grootste deel van de corpus
%% van je bachelorproef in dit hoofstuk verwerkt! Dit hoofdstuk is eerder een
%% kort overzicht van je plan van aanpak.
%%
%% Maak voor elke fase (behalve het literatuuronderzoek) een NIEUW HOOFDSTUK aan
%% en geef het een gepaste titel.

\lipsum[21-25]



% Voeg hier je eigen hoofdstukken toe die de ``corpus'' van je bachelorproef
% vormen. De structuur en titels hangen af van je eigen onderzoek. Je kan bv.
% elke fase in je onderzoek in een apart hoofdstuk bespreken.

%\input{...}
%\input{...}
%...

%%=============================================================================
%% Conclusie
%%=============================================================================

\chapter{Conclusie}%
\label{ch:conclusie}

% TODO: Trek een duidelijke conclusie, in de vorm van een antwoord op de
% onderzoeksvra(a)g(en). Wat was jouw bijdrage aan het onderzoeksdomein en
% hoe biedt dit meerwaarde aan het vakgebied/doelgroep? 
% Reflecteer kritisch over het resultaat. In Engelse teksten wordt deze sectie
% ``Discussion'' genoemd. Had je deze uitkomst verwacht? Zijn er zaken die nog
% niet duidelijk zijn?
% Heeft het onderzoek geleid tot nieuwe vragen die uitnodigen tot verder 
%onderzoek?

% \lipsum[76-80]
De conclusie wordt uitgewerkt in een volgende versie van de BP.



%---------- Bijlagen -----------------------------------------------------------

\appendix

\chapter{Onderzoeksvoorstel}

Het onderwerp van deze bachelorproef is gebaseerd op een onderzoeksvoorstel dat vooraf werd beoordeeld door de promotor. Dat voorstel is opgenomen in deze bijlage.

%% TODO: 
%\section*{Samenvatting}

% Kopieer en plak hier de samenvatting (abstract) van je onderzoeksvoorstel.
Geospatiale remote sensing biedt innovatieve methoden voor het monitoren van veranderingen in het landschap. 
Dit onderzoek richt zich specifiek op de detectie en visualisatie van veranderingen in vegetatiebedekking in de stad Gent tussen 2020 en 2024.
Hiervoor worden Sentinel-2 satellietbeelden gebruikt en geanalyseerd met behulp van machine learning-technieken. 
Een geavanceerde machine learning-pijplijn wordt ontwikkeld om deze veranderingen te detecteren en te interpreteren.
De onderzoeksmethode omvat een vergelijkende studie van diverse algoritmen, waaronder Support Vector Machines (SVM), Random Forests, 
Convolutional Neural Networks (CNN), Siamese Networks en Vision Transformers. Preprocessing van de data, 
zoals herschalen en semantische segmentatie, zorgt ervoor dat de inputdata consistent en geschikt is voor training. 
De algoritmen worden beoordeeld op nauwkeurigheid, consistentie en vermogen om visueel interpreteerbare resultaten te genereren.
Een belangrijke evaluatie metriek is de Vegetation Condition Index (VCI), die veranderingen in vegetatiebedekking kwantificeert door 
Normalized Difference Vegetation Index (NDVI)-waarden te vergelijken. Daarnaast worden metrieken zoals precision, recall en 
Intersection over Union (IoU) toegepast om de prestaties van de modellen te beoordelen. Het onderzoek beoogt niet alleen inzicht te bieden 
in de geschiktheid van verschillende machine learning-modellen voor geospatiale change detection, maar ook bij te dragen aan de 
toepassing van AI in stedelijke milieuanalyse en besluitvorming. Het onderzoek fungeert ook als oproep tot actie voor natuurbehoud in steden.

% Verwijzing naar het bestand met de inhoud van het onderzoeksvoorstel
%---------- Inleiding ---------------------------------------------------------

\section{Introductie}%
\label{sec:introductie}

%Waarover zal je bachelorproef gaan? Introduceer het thema en zorg dat volgende zaken zeker duidelijk aanwezig zijn:

%\begin{itemize}
%  \item kaderen thema
%  \item de doelgroep
%  \item de probleemstelling en (centrale) onderzoeksvraag
%  \item de onderzoeksdoelstelling
%\end{itemize}

%Denk er aan: een typische bachelorproef is \textit{toegepast onderzoek}, wat betekent dat je start vanuit een concrete probleemsituatie in bedrijfscontext, een \textbf{casus}. Het is belangrijk om je onderwerp goed af te bakenen: je gaat voor die \textit{ene specifieke probleemsituatie} op zoek naar een goede oplossing, op basis van de huidige kennis in het vakgebied.

%De doelgroep moet ook concreet en duidelijk zijn, dus geen algemene of vaag gedefinieerde groepen zoals \emph{bedrijven}, \emph{developers}, \emph{Vlamingen}, enz. Je richt je in elk geval op it-professionals, een bachelorproef is geen populariserende tekst. Eén specifiek bedrijf (die te maken hebben met een concrete probleemsituatie) is dus beter dan \emph{bedrijven} in het algemeen.

%Formuleer duidelijk de onderzoeksvraag! De begeleiders lezen nog steeds te veel voorstellen waarin we geen onderzoeksvraag terugvinden.

%Schrijf ook iets over de doelstelling. Wat zie je als het concrete eindresultaat van je onderzoek, naast de uitgeschreven scriptie? Is het een proof-of-concept, een rapport met aanbevelingen, \ldots Met welk eindresultaat kan je je bachelorproef als een succes beschouwen?
\textcolor{hogent-purple}{Geo-ICT is de toepassing van informatie- en communicatietechnologie in de geografie. Een belangrijk component van Geo-ICT is het programma GIS.
Geografische informatiesystemen (GIS) zijn digitale systemen die geografische gegevens analyseren, visualiseren en beheren om ruimtelijke 
patronen en relaties te begrijpen. Het wordt vandaag de dag door bedrijven gebruikt in verschillende sectoren. 
Nu er vraag is naar GIS-applicaties, zijn er ook ondernemingen die zich specialiseren in deze trend, zoals bijvoorbeeld het bedrijf GeoAI. 
Zij combineren Geo-ICT met het experimentele veld van machine learning. Dit kan ingezet worden om verschillende doeleinden te bereiken, zoals
bijvoorbeeld automatische kaartgeneratie, voorspellende modellering, \ldots.}
\newline
Het thema waar ik me op focus is geospatiale remote sensing change detection. Dit gaat over het ontdekken van veranderingen in een bepaald landschap.
Hier komt de sterkte van AI naar boven. Het is in staat patronen te ontdekken die wij als mensen moeilijk begrijpen.
Deze studie is gericht op het bestuderen van de aanwezigheid van vegetatie in de stad Gent tussen de periode van 2020-2024. 
De dataset bestaat uit afbeeldingen genomen door de Sentinel 2 satelliet. Tijdens het onderzoek ga ik verschillende machine learning 
algoritmes (Support Vector Machines (SVM), Random Forests, Convolutional Neural Networks (CNNs), Siamese Networks en Vision transformer) 
testen op de dataset. Het model moet in staat zijn vegetatie te detecteren en ook het verschil tussen de twee periodes te visualiseren. 
Het open-sourceplatform QGIS wordt gebruikt om de data te verkrijgen en te visualiseren.
\newline
\textcolor{hogent-yellow}{De onderzoeksvraag: Welk machine learning-model is het meest geschikt voor het nauwkeurig detecteren en visualiseren van de
veranderingen in aanwezigheid van vegetatie in de stad Gent tussen 2020 en 2024, met behulp van remote sensing change detection?}
\newline
\textcolor{hogent-darkgreen}{De deelvragen:
\newline
- Welke object klasse (vegetatie, bebouwing, wegen enz \ldots) toont de grootste veranderingen in percentage tussen de twee tijdstippen?
\newline
- Zijn er specifieke patronen in de veranderingen, zoals groei, afname of stabiliteit van groen in bepaalde delen van de stad?
\newline
- Zijn er ruimtelijke patronen in de veranderingen van vegetatie in de stad Gent?
\newline
- Is er een verschil tussen handmatige feature-extractie en automatische extractie op basis van deep learning, en wat is de impact hiervan op de resultaten?
\newline
- Welke evaluatie methoden zijn het meest geschikt om de prestaties van het model te beoordelen?
\newline
- Hoe wordt de balans gemeten tussen nauwkeurigheid en andere prestatie-indicatoren, zoals recall of precision?
\newline
- Welke criteria bepalen of een model als `acceptabel' wordt beschouwd?
\newline
- Is één dataset voldoende om de generaliseerbaarheid van het model te garanderen?
\newline
- Hoe wordt omgegaan met mogelijke bias of beperkingen in de gebruikte datasets?
\newline
- Welke preprocessing-stappen zijn nodig om de datasets geschikt te maken voor model input?}



%---------- Stand van zaken ---------------------------------------------------

\section{State-of-the-art}%
\label{sec:state-of-the-art}

% Hier beschrijf je de \emph{state-of-the-art} rondom je gekozen onderzoeksdomein, d.w.z.\ een inleidende, doorlopende tekst over het onderzoeksdomein van je bachelorproef. Je steunt daarbij heel sterk op de professionele \emph{vakliteratuur}, en niet zozeer op populariserende teksten voor een breed publiek. Wat is de huidige stand van zaken in dit domein, en wat zijn nog eventuele open vragen (die misschien de aanleiding waren tot je onderzoeksvraag!)?

% Je mag de titel van deze sectie ook aanpassen (literatuurstudie, stand van zaken, enz.). Zijn er al gelijkaardige onderzoeken gevoerd? Wat concluderen ze? Wat is het verschil met jouw onderzoek?

% Verwijs bij elke introductie van een term of bewering over het domein naar de vakliteratuur, bijvoorbeeld~\autocite{Hykes2013}! Denk zeker goed na welke werken je refereert en waarom.

% Draag zorg voor correcte literatuurverwijzingen! Een bronvermelding hoort thuis \emph{binnen} de zin waar je je op die bron baseert, dus niet er buiten! Maak meteen een verwijzing als je gebruik maakt van een bron. Doe dit dus \emph{niet} aan het einde van een lange paragraaf. Baseer nooit teveel aansluitende tekst op eenzelfde bron.

% Als je informatie over bronnen verzamelt in JabRef, zorg er dan voor dat alle nodige info aanwezig is om de bron terug te vinden (zoals uitvoerig besproken in de lessen Research Methods).

% Change detecion
Change detection (CD) is het waarnemen van verandering in een bepaald gebied aan de hand van afbeeldingen genomen op verschillende tijden 
\autocite{SINGH_1989}. Machine learning wordt al geruime tijd toegepast op CD. Maar in de laatste jaren heeft de opkomst van verschillende 
nieuwere algoritmen het veld verbreed. Zo worden verschillende studies gedaan rond de impact van deep learning neurale netwerken op CD,
zoals het onderzoek van \textcite{Bai_2022} over Deep Learning Change Detection (DLCD). Deep learning brengt een groot voordeel met zich 
mee, namelijk dat het in staat is om automatische feature extractie toe te passen op ruwe data zoals bewezen in de studie van \textcite{LeCun_2015}.
De meest voorkomende algoritmen zijn de CNN, Vision transformer en Siamese neurale netwerken. 
% Siamese networks
Siamese networks zijn voor het eerst voorgesteld in het onderzoek van \textcite{NIPS1993_288cc0ff}. Het is een algoritme dat bestaat uit
twee of meerdere identieke neurale netwerken met dezelfde weights en biases die samenkomen tot één output \autocite{Serrano_2023}. 
Een van de grootste voordelen ligt in het feit dat het in staat is accurate predicties te maken met weinig input data. 
Het wordt daarom ook wel het one-shot model genoemd \autocite{koch2015siamese}. 
% CNN
Convolutional neurale netwerken worden gezien als één van de beste machine learning algoritmes als het aankomt op visualisatietaken. De afbeelding wordt opgedeeld 
in kleine deeltjes aan de hand van filters, die dan allemaal geanalyseerd worden op een kleinere schaal \autocite{Geron2022}. Dit maakt het ideaal om hierarchische 
features en patronen te ontdekken in de input data.
% Vision Transformer
Transformers zijn neurale netwerken die uitsluitend bestaan uit attention layers voorgesteld in het onderzoek van \textcite{Vaswani2017}.
Attention is een methode om alleen relevante input data te verwerken in plaats van de hele batch \autocite{Geron2022}. 
Dit vermindert de trainingstijd zonder de nauwkeurigheid van het model significant te beïnvloeden. Vision transformer (ViT) is 
een transformer gebouwd voor visuele taken. Oorspronkelijk geïntroduceerd in de studie van \textcite{dosovitskiy2020image}. 
Vision Transformers (ViTs) zijn, dankzij de transformer-architectuur, bijzonder goed afgestemd op grootschalige datasets, 
wat bijdraagt aan hun huidige populariteit.
% Non Deep Learning modellen
Buiten deep learning zijn er ook nog andere AI modellen die gebruikt worden voor CD.
Deze studie focust op twee modellen: Support Vector Machines (SVM) en Random Forests.
% SVM
SVM's, oorspronkelijk geïntroduceerd in het onderzoek van \textcite{cortes1995support}, maken gebruik van feature extraction om zo een 
decision boundary (hyperlane) te bouwen die de veranderingen in het gebied scheidt van het onveranderde deel. 
Er zijn al verschillende studies gedaan rondom het gebruik van support vector machines voor CD, zoals \textcite{bovolo2008novel} en \textcite{Habib_2009}
% Random forests
Random forests is een ensemble learning model dat bestaat uit meerdere decision trees, 
waarbij elke tree gebaseerd is op willekeurige vectoren met dezelfde verdeling \autocite{Breiman_2001}. Meerdere studies zoals 
\textcite{Wessels_2016} en \textcite{Feng_2018} tonen aan dat random Forest geschikt is voor change detection.


% Voor literatuurverwijzingen zijn er twee belangrijke commando's:
% \autocite{KEY} => (Auteur, jaartal) Gebruik dit als de naam van de auteur
%   geen onderdeel is van de zin.
% \textcite{KEY} => Auteur (jaartal)  Gebruik dit als de auteursnaam wel een
%   functie heeft in de zin (bv. ``Uit onderzoek door Doll & Hill (1954) bleek
%   ...'')

%---------- Methodologie ------------------------------------------------------
\section{Methodologie}%
\label{sec:methodologie}

% Hier beschrijf je hoe je van plan bent het onderzoek te voeren. Welke onderzoekstechniek ga je toepassen om elk van je onderzoeksvragen te beantwoorden? Gebruik je hiervoor literatuurstudie, interviews met belanghebbenden (bv.~voor requirements-analyse), experimenten, simulaties, vergelijkende studie, risico-analyse, PoC, \ldots?

% Valt je onderwerp onder één van de typische soorten bachelorproeven die besproken zijn in de lessen Research Methods (bv.\ vergelijkende studie of risico-analyse)? Zorg er dan ook voor dat we duidelijk de verschillende stappen terug vinden die we verwachten in dit soort onderzoek!

% Vermijd onderzoekstechnieken die geen objectieve, meetbare resultaten kunnen opleveren. Enquêtes, bijvoorbeeld, zijn voor een bachelorproef informatica meestal \textbf{niet geschikt}. De antwoorden zijn eerder meningen dan feiten en in de praktijk blijkt het ook bijzonder moeilijk om voldoende respondenten te vinden. Studenten die een enquête willen voeren, hebben meestal ook geen goede definitie van de populatie, waardoor ook niet kan aangetoond worden dat eventuele resultaten representatief zijn.

% Uit dit onderdeel moet duidelijk naar voor komen dat je bachelorproef ook technisch voldoen\-de diepgang zal bevatten. Het zou niet kloppen als een bachelorproef informatica ook door bv.\ een student marketing zou kunnen uitgevoerd worden.

% Je beschrijft ook al welke tools (hardware, software, diensten, \ldots) je denkt hiervoor te gebruiken of te ontwikkelen.

% Probeer ook een tijdschatting te maken. Hoe lang zal je met elke fase van je onderzoek bezig zijn en wat zijn de concrete \emph{deliverables} in elke fase?


Het uiteindelijke doel van dit onderzoek is het ontwikkelen van een machine learning-pijplijn die effectief veranderingen 
kan detecteren tussen de aanwezigheid van vegetatie tussen twee tijdstippen. Dit is een vergelijkende studie, 
waarbij verschillende machine learning modellen getest worden. 
De eerste taak is het samenstellen van twee datasets opgebouwd uit satellietfoto's van Sentinel 2. 
Dataset 1 zal bestaan uit afbeeldingen van Gent in 2020 (t₁).  
De tweede dataset bestaat uit satellietfoto's van Gent in 2024 (t₂). 
Hiermee gaan we veranderingen proberen waar te nemen door ze te vergelijken met de afbeeldingen van t₁ met behulp van machine learning.
De datasets zijn opgedeeld in drie delen: trainging (60\%), validatie (20\%) en testing (20\%).
Training om het model te trainen. Validatie om te controleren hoe goed het model werkt op data die het niet eerder heeft gezien. 
Dit zorgt ervoor dat het model beter generialiseerd tijdens de testing fase. Testing om het model te evalueren. 
Bovendien moet de dataset goed gerepresenteerd zijn, zodat de accuraatheid van het model niet lijdt onder gelijkaardige situaties 
(het model moet in staat zijn de objecten te detecteren tijdens de verschillende seizoenen).
Elk vooraf gedefinieerd model wordt getraind en getest op de data. Het is niet de bedoeling om ze vanaf nul op te bouwen. Er wordt 
gebruik gemaakt van pretrained modellen. Dit deel van de studie zal herhaaldelijk worden toegepast op de dataset. Eerst passen we 
preprocessing toe op de datasets. In dit geval wordt herschalen toegepast om alle afbeeldingen dezelfde grootte te geven (Data Augmentatie). 
De data wordt gelabeld aan de hand van semantische segmentatie. Een computer vision taak waarbij elke pixel tot een bepaalde klasse/object wordt 
geclassifiseerd. Dit resulteerd in een segmentatie kaart waarin elk object (bv. vegetatie, gebouw, straat enz.) wordt aangeduid. Objecten 
van t₁ worden vergeleken met objecten van t₂. De veranderingen worden gedetecteerd door gemiddelde kleur, grootte en 
verschil in vorm van een object. Vervolgens worden de verschillen tussen t₁ en t₂ geclassifiseerd. Hierbij wordt niet alleen 
verandering aangeduid maar ook waarin objecten zijn veranderd. Voor de evaluatie worden een aantal metrieken gebruikt. De belangrijkste 
hieronder is de vegetation condition index (VCI). De VCI vergelijkt de NDVI-waarden van twee tijdstippen (t₁ en t₂) om veranderingen 
in vegetatiebedekking te berekenen. Andere evaluatie methoden zoals confusiematrix-methoden (precision en recall) en IoU (Intersection over Union)
worden ook gebruikt. De twee belangrijkste python machine learning frameworks die gebruikt zullen worden tijdens deze studie zijn TensorFlow en scikit-learn.


%---------- Verwachte resultaten ----------------------------------------------
\section{Verwacht resultaat, conclusie}%
\label{sec:verwachte_resultaten}

% Hier beschrijf je welke resultaten je verwacht. Als je metingen en simulaties uitvoert, kan je hier al mock-ups maken van de grafieken samen met de verwachte conclusies. Benoem zeker al je assen en de onderdelen van de grafiek die je gaat gebruiken. Dit zorgt ervoor dat je concreet weet welk soort data je moet verzamelen en hoe je die moet meten.

% Wat heeft de doelgroep van je onderzoek aan het resultaat? Op welke manier zorgt jouw bachelorproef voor een meerwaarde?

% Hier beschrijf je wat je verwacht uit je onderzoek, met de motivatie waarom. Het is \textbf{niet} erg indien uit je onderzoek andere resultaten en conclusies vloeien dan dat je hier beschrijft: het is dan juist interessant om te onderzoeken waarom jouw hypothesen niet overeenkomen met de resultaten.

Voor het technische aspect van deze studie wordt verwacht dat de deep learning modellen, zoals Convolutional Neural Networks (CNN), 
Vision Transformers en Siamese netwerken, betere prestaties zullen leveren in vergelijking met Support Vector Machines (SVM) en 
Random Forests. Recente onderzoeken ondersteunen deze verwachting door aan te tonen dat deep learning algoritmen doorgaans 
superieure resultaten behalen, met name bij complexe taken zoals geospatiale veranderingendetectie.
\newline
Met betrekking tot de veranderingen in vegetatiebedekking binnen Gent wordt een lichte toename verwacht in vergelijking met 2020, 
gebaseerd op trends in stedelijke vergroening en gerelateerde ecologische initiatieven in de regio.
\newline
Dit onderzoek gaat over het detecteren van veranderingen in de vegetatie in een stedelijk gebied. Het is gericht naar organisaties 
die het behoud van de natuur op zich nemen. Zo kunnen rapporten met zulke studies een call to action worden voor het verder 
behouden van onze natuur.

%%---------- Andere bijlagen --------------------------------------------------
% TODO: Voeg hier eventuele andere bijlagen toe. Bv. als je deze BP voor de
% tweede keer indient, een overzicht van de verbeteringen t.o.v. het origineel.
%\input{...}

%%---------- Backmatter, referentielijst ---------------------------------------

\backmatter{}

\setlength\bibitemsep{2pt} %% Add Some space between the bibliograpy entries
\printbibliography[heading=bibintoc]

\end{document}
