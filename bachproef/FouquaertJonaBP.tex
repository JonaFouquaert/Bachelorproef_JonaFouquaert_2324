%===============================================================================
% LaTeX sjabloon voor de bachelorproef toegepaste informatica aan HOGENT
% Meer info op https://github.com/HoGentTIN/latex-hogent-report
%===============================================================================

\documentclass[dutch,dit,thesis]{hogentreport}

% TODO:
% - If necessary, replace the option `dit`' with your own department!
%   Valid entries are dbo, dbt, dgz, dit, dlo, dog, dsa, soa
% - If you write your thesis in English (remark: only possible after getting
%   explicit approval!), remove the option "dutch," or replace with "english".

\usepackage{lipsum} % For blind text, can be removed after adding actual content

%% Pictures to include in the text can be put in the graphics/ folder
\graphicspath{{graphics/}}

%% For source code highlighting, requires pygments to be installed
%% Compile with the -shell-escape flag!
\usepackage[section]{minted}
%% If you compile with the make_thesis.{bat,sh} script, use the following
%% import instead:
%% \usepackage[section,outputdir=../output]{minted}
\usemintedstyle{solarized-light}
\definecolor{bg}{RGB}{253,246,227} %% Set the background color of the codeframe

%% Change this line to edit the line numbering style:
\renewcommand{\theFancyVerbLine}{\ttfamily\scriptsize\arabic{FancyVerbLine}}

%% Macro definition to load external java source files with \javacode{filename}:
\newmintedfile[javacode]{java}{
    bgcolor=bg,
    fontfamily=tt,
    linenos=true,
    numberblanklines=true,
    numbersep=5pt,
    gobble=0,
    framesep=2mm,
    funcnamehighlighting=true,
    tabsize=4,
    obeytabs=false,
    breaklines=true,
    mathescape=false
    samepage=false,
    showspaces=false,
    showtabs =false,
    texcl=false,
}

% Other packages not already included can be imported here

%%---------- Document metadata -------------------------------------------------
% TODO: Replace this with your own information
\author{Jona Fouquaert}
\supervisor{Dhr. S. De Gheselle}
\cosupervisor{Dhr. C. Stal}
\title{Diepgaande Analyse van Geospatiale Veranderingen: Een Vergelijkende Studie van Machine Learning Modellen in Remote Sensing Change Detection voor de Vegetatie van Gent (2020-2024)}
\academicyear{\advance\year by -1 \the\year--\advance\year by 1 \the\year}
\examperiod{1}
\degreesought{\IfLanguageName{dutch}{Professionele bachelor in de toegepaste informatica}{Bachelor of applied computer science}}
\partialthesis{false} %% To display 'in partial fulfilment'
%\institution{Internshipcompany BVBA.}

%% Add global exceptions to the hyphenation here
\hyphenation{back-slash}

%% The bibliography (style and settings are  found in hogentthesis.cls)
\addbibresource{bachproef.bib}            %% Bibliography file
\addbibresource{../voorstel/voorstel.bib} %% Bibliography research proposal
\defbibheading{bibempty}{}

%% Prevent empty pages for right-handed chapter starts in twoside mode
\renewcommand{\cleardoublepage}{\clearpage}

\renewcommand{\arraystretch}{1.2}

%% Content starts here.
\begin{document}

%---------- Front matter -------------------------------------------------------

\frontmatter

\hypersetup{pageanchor=false} %% Disable page numbering references
%% Render a Dutch outer title page if the main language is English
\IfLanguageName{english}{%
    %% If necessary, information can be changed here
    \degreesought{Professionele Bachelor toegepaste informatica}%
    \begin{otherlanguage}{dutch}%
       \maketitle%
    \end{otherlanguage}%
}{}

%% Generates title page content
\maketitle
\hypersetup{pageanchor=true}

\input{voorwoord}
%%=============================================================================
%% Samenvatting
%%=============================================================================

% TODO: De "abstract" of samenvatting is een kernachtige (~ 1 blz. voor een
% thesis) synthese van het document.
%
% Een goede abstract biedt een kernachtig antwoord op volgende vragen:
%
% 1. Waarover gaat de bachelorproef?
% 2. Waarom heb je er over geschreven?
% 3. Hoe heb je het onderzoek uitgevoerd?
% 4. Wat waren de resultaten? Wat blijkt uit je onderzoek?
% 5. Wat betekenen je resultaten? Wat is de relevantie voor het werkveld?
%
% Daarom bestaat een abstract uit volgende componenten:
%
% - inleiding + kaderen thema
% - probleemstelling
% - (centrale) onderzoeksvraag
% - onderzoeksdoelstelling
% - methodologie
% - resultaten (beperk tot de belangrijkste, relevant voor de onderzoeksvraag)
% - conclusies, aanbevelingen, beperkingen
%
% LET OP! Een samenvatting is GEEN voorwoord!

%%---------- Nederlandse samenvatting -----------------------------------------
%
% TODO: Als je je bachelorproef in het Engels schrijft, moet je eerst een
% Nederlandse samenvatting invoegen. Haal daarvoor onderstaande code uit
% commentaar.
% Wie zijn bachelorproef in het Nederlands schrijft, kan dit negeren, de inhoud
% wordt niet in het document ingevoegd.

% \IfLanguageName{english}{%
% \selectlanguage{dutch}
% \chapter*{Samenvatting}
% \lipsum[1-4]
% \selectlanguage{english}
% }{}

%%---------- Samenvatting -----------------------------------------------------
% De samenvatting in de hoofdtaal van het document

\chapter*{\IfLanguageName{dutch}{Samenvatting}{Abstract}}

Het abstract wordt uitgewerkt in een latere versie van de BP

% Geospatiale remote sensing biedt innovatieve methoden voor het monitoren van veranderingen in het landschap. 
% Dit onderzoek richt zich specifiek op de detectie en visualisatie van veranderingen in vegetatiebedekking in de stad Gent tussen 2020 en 2024.
% Hiervoor worden Sentinel-2 satellietbeelden gebruikt en geanalyseerd met behulp van machine learning-technieken. 
% Een geavanceerde machine learning-pijplijn wordt ontwikkeld om deze veranderingen te detecteren en te interpreteren.
% De onderzoeksmethode omvat een vergelijkende studie van diverse algoritmen, waaronder Support Vector Machines (SVM), Random Forests, 
% Convolutional Neural Networks (CNN), Siamese Networks en Vision Transformers. Preprocessing van de data, 
% zoals herschalen en semantische segmentatie, zorgt ervoor dat de inputdata consistent en geschikt is voor training. 
% De algoritmen worden beoordeeld op nauwkeurigheid, consistentie en vermogen om visueel interpreteerbare resultaten te genereren.
% Een belangrijke evaluatie metriek is de Vegetation Condition Index (VCI), die veranderingen in vegetatiebedekking kwantificeert door 
% Normalized Difference Vegetation Index (NDVI)-waarden te vergelijken. Daarnaast worden metrieken zoals precision, recall en 
% Intersection over Union (IoU) toegepast om de prestaties van de modellen te beoordelen. Het onderzoek beoogt niet alleen inzicht te bieden 
% in de geschiktheid van verschillende machine learning-modellen voor geospatiale change detection, maar ook bij te dragen aan de 
% toepassing van AI in stedelijke milieuanalyse en besluitvorming. Het onderzoek fungeert ook als oproep tot actie voor natuurbehoud in steden.


%---------- Inhoud, lijst figuren, ... -----------------------------------------

\tableofcontents

% In a list of figures, the complete caption will be included. To prevent this,
% ALWAYS add a short description in the caption!
%
%  \caption[short description]{elaborate description}
%
% If you do, only the short description will be used in the list of figures

\listoffigures

% If you included tables and/or source code listings, uncomment the appropriate
% lines.
%\listoftables
%\listoflistings

% Als je een lijst van afkortingen of termen wil toevoegen, dan hoort die
% hier thuis. Gebruik bijvoorbeeld de ``glossaries'' package.
% https://www.overleaf.com/learn/latex/Glossaries

%---------- Kern ---------------------------------------------------------------

\mainmatter{}

% De eerste hoofdstukken van een bachelorproef zijn meestal een inleiding op
% het onderwerp, literatuurstudie en verantwoording methodologie.
% Aarzel niet om een meer beschrijvende titel aan deze hoofdstukken te geven of
% om bijvoorbeeld de inleiding en/of stand van zaken over meerdere hoofdstukken
% te verspreiden!

%%=============================================================================
%% Inleiding
%%=============================================================================

\chapter{\IfLanguageName{dutch}{Inleiding}{Introduction}}%
\label{ch:inleiding}

% De inleiding moet de lezer net genoeg informatie verschaffen om het onderwerp te begrijpen en in te zien waarom de onderzoeksvraag de moeite waard is om te onderzoeken. In de inleiding ga je literatuurverwijzingen beperken, zodat de tekst vlot leesbaar blijft. Je kan de inleiding verder onderverdelen in secties als dit de tekst verduidelijkt. Zaken die aan bod kunnen komen in de inleiding~\autocite{Pollefliet2011}:

% \begin{itemize}
%   \item context, achtergrond
%   \item afbakenen van het onderwerp
%   \item verantwoording van het onderwerp, methodologie
%   \item probleemstelling
%   \item onderzoeksdoelstelling
%   \item onderzoeksvraag
%   \item \ldots
% \end{itemize}

% Geo-ICT is de toepassing van informatie- en communicatietechnologie in de geografie. Een belangrijk component van Geo-ICT is het programma GIS.
% Geografische informatiesystemen (GIS) zijn digitale systemen die geografische gegevens analyseren, visualiseren en beheren om ruimtelijke 
% patronen en relaties te begrijpen. Het wordt vandaag de dag door bedrijven gebruikt in verschillende sectoren. 
% Nu er vraag is naar GIS-applicaties, zijn er ook ondernemingen die zich specialiseren in deze trend, zoals bijvoorbeeld het bedrijf GeoAI. 
% Zij combineren Geo-ICT met het experimentele veld van machine learning. Dit kan ingezet worden om verschillende doeleinden te bereiken, zoals
% bijvoorbeeld automatische kaartgeneratie, voorspellende modellering, enz\ldots.
% \newline
% Het thema waar ik me op focus is geospatiale remote sensing change detection. Dit richt zich op het ontdekken van veranderingen in een bepaald landschap.
% Hier komt de sterkte van AI naar boven. Het is in staat patronen te ontdekken die wij als mensen moeilijk begrijpen.
% Deze studie is gericht op het bestuderen van de aanwezigheid van vegetatie in de stad Gent tussen de periode van 2020-2024. 
% De dataset bestaat uit luchtfoto's genomen door het agentschap digitaal Vlaanderen (middenschalige orthofotobedekking van het Vlaamse Gewest). 
% Tijdens het onderzoek worden verschillende machine learning algoritmes (Support Vector Machines (SVM), Random Forests, 
% Convolutional Neural Networks (CNNs), Siamese Networks en Vision transformer) getest op de dataset. 
% Het model moet in staat zijn vegetatie te detecteren en ook het verschil tussen de twee periodes te visualiseren. 
% Het open-sourceplatform QGIS wordt gebruikt om de data te verkrijgen en te visualiseren.

% Change detection (CD) is het waarnemen van verandering in een bepaald gebied aan de hand van afbeeldingen genomen op verschillende tijden. 
% Machine learning wordt al geruime tijd toegepast op CD. Vroeger lag de focus meer op tradionele pixel based methodes. 
% Maar in de laatste jaren heeft de opkomst van verschillende nieuwere algoritmen het veld verbreed. Zo worden verschillende studies gedaan 
% rond de impact van deep learning neurale netwerken op CD, zoals het onderzoek van Bai e.a. (2022) over Deep Learning Change Detection (DLCD).

\section{\IfLanguageName{dutch}{Probleemstelling}{Problem Statement}}%
\label{sec:probleemstelling}

% Uit je probleemstelling moet duidelijk zijn dat je onderzoek een meerwaarde heeft voor een concrete doelgroep. De doelgroep moet goed gedefinieerd en afgelijnd zijn. Doelgroepen als ``bedrijven,'' ``KMO's'', systeembeheerders, enz.~zijn nog te vaag. Als je een lijstje kan maken van de personen/organisaties die een meerwaarde zullen vinden in deze bachelorproef (dit is eigenlijk je steekproefkader), dan is dat een indicatie dat de doelgroep goed gedefinieerd is. Dit kan een enkel bedrijf zijn of zelfs één persoon (je co-promotor/opdrachtgever).

Klimaatverandering is onbetwistbaar het centrale thema van de 21e eeuw, met de opwarming van de aarde die een scala aan negatieve 
effecten met zich meebrengt, waaronder stijgende zeespiegels en drogere klimaten. Deze droogte draagt bij aan een toename van 
bosbranden wereldwijd, wat des te zorgwekkender is gezien de cruciale rol die bossen spelen bij het tegengaan van klimaatverandering. 
Tegelijkertijd worden we geconfronteerd met een groeiende overbevolking, wat resulteert in een toenemende behoefte aan huisvesting 
en de daaropvolgende afname van natuurlijke habitat ‘s wereldwijd. Het is daarom van essentieel belang om stedelijke ontwikkeling en behoud 
van natuurlijke omgevingen hand in hand te laten gaan.
\newline
\newline
In 2022 stelde de Europese Commissie de natuurherstel wet voor. Die verplicht lidstaten om tegen 2030 tot 30 procent van 
verwaarloosde ecosystemen te herstellen. De wet werd uiteindelijk goedgekeurd op 17 juni 2024. Een bijkomende maatregel in dit besluit is een
oproep naar meer groene ruimtes in steden.
Verschillende onderzoeken zoals \textcite{bajirao2015importance} en \textcite{Birch_2020} tonen aan dat een groene omgeving een positief effect heeft op 
menselijk welzijn. Tijdens de zomermaanden zorgt stedelijke infrastructuur voor 
hogere temperaturen. In contrast blijkt dat vegetatie juist tot een lagere omgevingstemperatuur leidt \autocite{vuckovic2017studies}.
In België en meer specifiek Vlaanderen is er ook een groot grondwater probleem. Door de verharding van de grond sijpelt regenwater moeilijker door. 
Dit heeft negatieve gevolgen zoals mogelijke overstromingen. Meer vegetatie zorgt voor betere grondwater doorsijpeling. 
Vegetatie bevordert ook de zuivering van de lucht. Luchtvervuiling is ondanks lage-emissiezones in steden, nog altijd een probleem.
Meer vegetatie betekend een betere luchtkwaliteit.
\newline
\newline
Deze bevindingen onderstrepen het belang van het integreren van groene ruimtes in stadsplanning en ontwikkeling. 
In lijn hiermee richt deze studie zich op het observeren van de ontwikkeling van vegetatie in stedelijke omgevingen, met als doel inzicht 
te krijgen in hoe deze omgevingen evolueren en hoe ze kunnen worden verbeterd ten behoeve van zowel de menselijke gezondheid als de natuur.
Dit onderzoek is gericht naar steden en gemeentes om een beter inzicht te krijgen wat voor impact beleid heeft op de natuur.
\newline
\newline

% Klimaatverandering is onbetwistbaar het centrale thema van de 21e eeuw, met de opwarming van de aarde die een scala aan negatieve 
% effecten met zich meebrengt, waaronder stijgende zeespiegels en drogere klimaten. Deze droogte draagt bij aan een toename van 
% bosbranden wereldwijd, wat des te zorgwekkender is gezien de cruciale rol die bossen spelen bij het tegengaan van klimaatverandering. 
% Tegelijkertijd worden we geconfronteerd met een groeiende overpopulatie, wat resulteert in een toenemende behoefte aan huisvesting 
% en de daaropvolgende afname van natuurlijke habitats wereldwijd. Het is daarom van essentieel belang om stedelijke ontwikkeling en behoud 
% van natuurlijke omgevingen hand in hand te laten gaan.
% \newline
% \newline
% Bepaalde onderzoeken zoals \textcite{Birch_2020}, benadrukt de positieve invloed van natuurlijke omgevingen in stedelijke gebieden op 
% de mentale gezondheid van de bewoners. In België, en met name in Vlaanderen, speelt bovendien een ernstig probleem met betrekking tot grondwaterbeheer. 
% Door toenemende bodemverharding wordt de infiltratie van regenwater belemmerd, wat kan leiden tot negatieve gevolgen zoals een 
% verhoogd risico op overstromingen. Een toename van vegetatie bevordert echter de waterdoorlaatbaarheid van de bodem, waardoor deze 
% problematiek gedeeltelijk kan worden verzacht. Tegelijkertijd kan de aanwezigheid van stedelijke infrastructuur leiden tot verhoogde 
% temperaturen tijdens de zomermaanden. Daartegenover staat dat vegetatie doorgaans geassocieerd wordt met lagere omgevingstemperaturen 
% in vergelijking met stedelijke structuren \autocite{vuckovic2017studies}.
% \newline
% \newline
% Deze bevindingen onderstrepen het belang van het integreren van groene ruimtes in stadsplanning en ontwikkeling. 
% In lijn hiermee richt dit onderzoek zich op het observeren van de ontwikkeling van vegetatie in stedelijke omgevingen, met als doel inzicht 
% te krijgen in hoe deze omgevingen evolueren en hoe ze kunnen worden verbeterd ten behoeve van zowel de menselijke gezondheid als de natuur.

\section{\IfLanguageName{dutch}{Onderzoeksvraag}{Research question}}%
\label{sec:onderzoeksvraag}

% Wees zo concreet mogelijk bij het formuleren van je onderzoeksvraag. Een onderzoeksvraag is trouwens iets waar nog niemand op dit moment een antwoord heeft (voor zover je kan nagaan). Het opzoeken van bestaande informatie (bv. ``welke tools bestaan er voor deze toepassing?'') is dus geen onderzoeksvraag. Je kan de onderzoeksvraag verder specifiëren in deelvragen. Bv.~als je onderzoek gaat over performantiemetingen, dan 

% Deze studie is een verglijkende proef tussen verschillende machine learning modellen om veranderingen te detecteren aan vegetatie 
% in een stedelijke omgeving over een bepaalde tijdsperiode. De modellen zijn opgedeeld in twee groepen. De nieuwere deep learning modellen en de 
% origenele algoritmen.

Dit onderzoek richt zich op de vraag welk machine learning-model het meest geschikt is voor het nauwkeurig detecteren en visualiseren 
van veranderingen in de aanwezigheid van vegetatie in de stad Gent tussen 2020 en 2024, met behulp van remote sensing change detection.
\newline
Ondersteunend met deze hoofdvraag zoek ik naar antwoorden voor een aantal deelvragen. Welke objectklasse (vegetatie, bebouwing, wegen enz.) 
tonen de grootste veranderingen in percentage tussen de twee tijdstippen? Wat is het verschil tussen handmatige (traditionele ML) en 
automatische feature extractie (Deep Learning)? Wat voor preprocessing stappen zijn nodig voor een ideale dataset? Wat zijn de belangrijkste 
evaluatie criteria bij het beoordelen van een model? Is één dataset genoeg om tot een generaliseerbaar model te komen?

% Dit onderzoek richt zich op de vraag welk machine learning-model het meest geschikt is voor het nauwkeurig detecteren en visualiseren 
% van veranderingen in de aanwezigheid van vegetatie in de stad Gent tussen 2020 en 2024, met behulp van remote sensing change detection.
% Om deze hoofdvraag te beantwoorden, worden verschillende deelaspecten geanalyseerd. Allereerst wordt onderzocht welke objectklasse 
% (vegetatie, bebouwing of wegen) de grootste veranderingen vertoont in percentage tussen de twee onderzochte tijdstippen. 
% Daarnaast wordt nagegaan of er specifieke patronen in de vegetatieverandering te identificeren zijn, zoals groei, afname of stabiliteit 
% in bepaalde delen van de stad. Hierbij wordt tevens bekeken of er ruimtelijke structuren of clusters in deze veranderingen aanwezig zijn.
% \newline
% \newline
% Een ander belangrijk aspect van de studie is de vergelijking tussen handmatige feature-extractie en automatische extractie op basis van 
% deep learning, waarbij de impact van beide methoden op de uiteindelijke resultaten wordt geanalyseerd. Daarnaast wordt onderzocht welke 
% evaluatiemethoden het meest geschikt zijn om de prestaties van de modellen te beoordelen en hoe de balans tussen nauwkeurigheid en 
% andere prestatie-indicatoren, zoals recall en precision, wordt gemeten.
% \newline
% Verder wordt bekeken welke criteria bepalen of een model als `acceptabel' wordt beschouwd binnen de context van change detection en of 
% één dataset voldoende is om de generaliseerbaarheid van het model te waarborgen. Hierbij wordt ook aandacht besteed aan mogelijke bias 
% of beperkingen in de gebruikte datasets en hoe hiermee kan worden omgegaan. Tot slot wordt geanalyseerd welke preprocessing-stappen 
% nodig zijn om de datasets geschikt te maken voor input in de machine learning-modellen.
% \newline
% Door middel van deze deelvragen wordt een systematische benadering gehanteerd om de effectiviteit en toepasbaarheid van verschillende 
% machine learning-modellen voor geospatiale change detection te evalueren.



% De onderzoeksvraag: Welk machine learning-model is het meest geschikt voor het nauwkeurig detecteren en visualiseren van de
% veranderingen in aanwezigheid van vegetatie in de stad Gent tussen 2020 en 2024, met behulp van remote sensing change detection?
% \newline
% De deelvragen: 
% \newline
% - Welke object klasse (vegetatie, bebouwing, wegen enz \ldots) toont de grootste veranderingen in percentage tussen de twee tijdstippen?
% \newline
% - Zijn er specifieke patronen in de veranderingen, zoals groei, afname of stabiliteit van groen in bepaalde delen van de stad?
% \newline
% - Zijn er ruimtelijke patronen in de veranderingen van vegetatie in de stad Gent?
% \newline
% - Is er een verschil tussen handmatige feature-extractie en automatische extractie op basis van deep learning, en wat is de impact hiervan op de resultaten?
% \newline
% - Welke evaluatie methoden zijn het meest geschikt om de prestaties van het model te beoordelen?
% \newline
% - Hoe wordt de balans gemeten tussen nauwkeurigheid en andere prestatie-indicatoren, zoals recall of precision?
% \newline
% - Welke criteria bepalen of een model als `acceptabel' wordt beschouwd?
% \newline
% - Is één dataset voldoende om de generaliseerbaarheid van het model te garanderen?
% \newline
% - Hoe wordt omgegaan met mogelijke bias of beperkingen in de gebruikte datasets?
% \newline
% - Welke preprocessing-stappen zijn nodig om de datasets geschikt te maken voor model input?


\section{\IfLanguageName{dutch}{Onderzoeksdoelstelling}{Research objective}}%
\label{sec:onderzoeksdoelstelling}

% Wat is het beoogde resultaat van je bachelorproef? Wat zijn de criteria voor succes? Beschrijf die zo concreet mogelijk. Gaat het bv.\ om een proof-of-concept, een prototype, een verslag met aanbevelingen, een vergelijkende studie, enz.

% Het doel van deze studie is het beste model te selecteren voor een change detection architectuur om veranderingen in vegetatie te 
% detecteren in een stedelijke omgeving. De succescriteria is de accuraatheid van het model. De snelheid is ook belangrijk maar niet 
% doorslagevend.

Dit is een vergelijkende studie met als doel het beste model te selecteren voor een change detecion architectuur om veranderingen in
vegetatie te detecteren in een stedelijke omgeving. Het resultaat zal een machine learning pipeline zijn bestaande uit het best 
scorende model. 
\newline
% Voor de evaluatie van de verschillende machine learning-modellen worden meerdere prestatie-indicatoren gehanteerd. 
% De Vegetation Condition Index (VCI) wordt als primaire metriek gebruikt om veranderingen in vegetatiebedekking te kwantificeren door 
% de Normalized Difference Vegetation Index (NDVI)-waarden van t₁ en t₂ met elkaar te vergelijken. Daarnaast worden aanvullende 
% evaluatiemethoden toegepast, waaronder confusion matrix-gebaseerde metrieken zoals precision en recall, evenals Intersection over Union (IoU)
% om de nauwkeurigheid van de voorspelde veranderingen te beoordelen.

\section{\IfLanguageName{dutch}{Opzet van deze bachelorproef}{Structure of this bachelor thesis}}%
\label{sec:opzet-bachelorproef}

% Het is gebruikelijk aan het einde van de inleiding een overzicht te
% geven van de opbouw van de rest van de tekst. Deze sectie bevat al een aanzet
% die je kan aanvullen/aanpassen in functie van je eigen tekst.

De rest van deze bachelorproef is als volgt opgebouwd:
\newline
\newline
In Hoofdstuk~\ref{ch:stand-van-zaken} wordt een overzicht gegeven van de stand van zaken binnen het onderzoeksdomein, op basis van een literatuurstudie.

In Hoofdstuk~\ref{ch:methodologie} wordt de methodologie toegelicht en worden de gebruikte onderzoekstechnieken besproken om een antwoord te kunnen formuleren op de onderzoeksvragen.

In Hoofdstuk~\ref{ch:model-evaluatie} worden de verschillende evaluatie technieken besproken om de verandering tussen beide tijdstippen te bestuderen. 

In Hoofdstuk~\ref{ch:model-resultaten} worden de resultaten van alle modellen weergeven.

% TODO: Vul hier aan voor je eigen hoofstukken, één of twee zinnen per hoofdstuk

In Hoofdstuk~\ref{ch:conclusie}, tenslotte, wordt de conclusie gegeven en een antwoord geformuleerd op de onderzoeksvragen. Daarbij wordt ook een aanzet gegeven voor toekomstig onderzoek binnen dit domein.
\chapter{\IfLanguageName{dutch}{Stand van zaken}{State of the art}}%
\label{ch:stand-van-zaken}

% Tip: Begin elk hoofdstuk met een paragraaf inleiding die beschrijft hoe
% dit hoofdstuk past binnen het geheel van de bachelorproef. Geef in het
% bijzonder aan wat de link is met het vorige en volgende hoofdstuk.

% Pas na deze inleidende paragraaf komt de eerste sectiehoofding.

% Dit hoofdstuk bevat je literatuurstudie. De inhoud gaat verder op de inleiding, maar zal het onderwerp van de bachelorproef *diepgaand* uitspitten. De bedoeling is dat de lezer na lezing van dit hoofdstuk helemaal op de hoogte is van de huidige stand van zaken (state-of-the-art) in het onderzoeksdomein. Iemand die niet vertrouwd is met het onderwerp, weet nu voldoende om de rest van het verhaal te kunnen volgen, zonder dat die er nog andere informatie moet over opzoeken \autocite{Pollefliet2011}.

% Je verwijst bij elke bewering die je doet, vakterm die je introduceert, enz.\ naar je bronnen. In \LaTeX{} kan dat met het commando \texttt{$\backslash${textcite\{\}}} of \texttt{$\backslash${autocite\{\}}}. Als argument van het commando geef je de ``sleutel'' van een ``record'' in een bibliografische databank in het Bib\LaTeX{}-formaat (een tekstbestand). Als je expliciet naar de auteur verwijst in de zin (narratieve referentie), gebruik je \texttt{$\backslash${}textcite\{\}}. Soms is de auteursnaam niet expliciet een onderdeel van de zin, dan gebruik je \texttt{$\backslash${}autocite\{\}} (referentie tussen haakjes). Dit gebruik je bv.~bij een citaat, of om in het bijschrift van een overgenomen afbeelding, broncode, tabel, enz. te verwijzen naar de bron. In de volgende paragraaf een voorbeeld van elk.

% \textcite{Knuth1998} schreef een van de standaardwerken over sorteer- en zoekalgoritmen. Experten zijn het erover eens dat cloud computing een interessante opportuniteit vormen, zowel voor gebruikers als voor dienstverleners op vlak van informatietechnologie~\autocite{Creeger2009}.

% Let er ook op: het \texttt{cite}-commando voor de punt, dus binnen de zin. Je verwijst meteen naar een bron in de eerste zin die erop gebaseerd is, dus niet pas op het einde van een paragraaf.

% \lipsum[7-20]

% Change detection
\section{\IfLanguageName{dutch}{Remote Sensing Change Detection}{Remote Sensing Change Detection}}%
\label{sec:remote-sensing-change-detection}

Remote sensing change detection (RSCD) is het waarnemen van verandering in een bepaald gebied aan de hand van op afstand genomen afbeeldingen op
verschillende tijdstippen \autocite{SINGH_1989}. Data bestaat meestal uit lucht- of satellietfoto’s. Gebruikte applicaties zijn onder meer het 
monitoren van ontbossing of de impact van natuurrampen bestuderen.
\newline
CD kan onder menselijke toezicht gebeuren door de twee foto's naast elkaar te leggen en zo de verschillen te zoeken. 
Na de opkomst van computers werden verschillende nieuwe methodes uitgevonden. Single-pixel methoden zoals image difference \autocite{QUARMBY_1989} 
en image ratio \autocite{HOWARTH_1981}. Andere bekende methode zijn principle component analysis \autocite{Deng_2008} en change vector analysis \autocite{Carvalho_J_nior_2011}. 
Rond de jaren 90, na de opkomst van computer vision, begon er ook binnen RSCD use cases te onstaan. Dit wordt aangetoond in studies zoals
\textcite{levien1999machine} en \textcite{dai1999remotely}. 
Sindsdien is er een groepering van traditionele en AI gebaseerde technieken ontstaan. Over de jaren heen zijn computer vision 
modellen alsmaar complexer geworden. Verschillende recente studies zoals \textcite{zerrouki2019machine} tonen aan dat machine learning algoritmen 
beter en efficiënter zijn in RSCD dan de traditionele methodes. 
\newline
\newline

% Remote sensing change detection (CD) verwijst naar het proces van het identificeren en analyseren van veranderingen in een 
% specifiek gebied op basis van afbeeldingen die op verschillende tijdstippen zijn genomen \autocite{SINGH_1989}. 
% Landschappelijke veranderingen kunnen significante gevolgen hebben voor zowel het klimaat als de leefomstandigheden van mens en dier. 
% Het systematisch monitoren van deze veranderingen is van cruciaal belang om potentiële problemen tijdig te signaleren en gepaste 
% maatregelen te nemen. Dit onderzoek richt zich specifiek op de aanwezigheid van vegetatie in een stedelijke omgeving.
% Traditioneel kan CD worden uitgevoerd door visuele inspectie, waarbij beelden uit verschillende tijdstippen naast elkaar worden gelegd 
% om verschillen te detecteren. Uit diverse onderzoeken zoals \textcite{zerrouki2019machine} blijkt echter dat machine learning-algoritmen aanzienlijk effectiever en 
% efficiënter zijn in het geautomatiseerd herkennen van veranderingen in het landschap.
% \newline
% \newline

% Computer vision
\section{\IfLanguageName{dutch}{Computer Vision}{Computer Vision}}%
\label{sec:computer-vision}

Computer vision is een onderdeel van artificiële intelligentie (AI) dat zich richt op het werken met visuele data (foto's, video's enz\ldots) 
\autocite{ibm2025b}. Het bekendste voorbeeld hiervan is foto classificatie. Een algoritme dat een gegeven foto kan classificeren tot een 
bepaalde klasse. Lange tijd was, deze op eerste zicht simpele taak, niet op betrouwbare wijze uit te voeren door machine learning \autocite{Geron2022}.
Met de opkomst van deep learning (DL) en het convolutionele neurale netwerk (CNN) is er in de laatste decennia veel vooruitgang geboekt
in dit veld. De eerste grote doorbraak kwam in 2012 met de overwinning van AlexNet in de ImageNet-competitie \autocite{krizhevsky2012imagenet}. 
Door de beschikbaarheid van grootschalige datasets (zoals COCO en ImageNet), verhoogde rekenkracht en nieuwe architecturen 
blijft het veld constant evolueren.

% De methode die voor dit onderzoek gebruikt zal worden is semantische segmentatie.


% Computer vision is een subdiscipline binnen de artificiële intelligentie (AI) die zich richt op het automatisch interpreteren, 
% begrijpen en extraheren van informatie uit visuele input, zoals foto's, video's of beelden afkomstig van sensoren \autocite{ibm2025b}. 
% Het doel van computer vision is om computers in staat te stellen visuele data op een manier te analyseren die vergelijkbaar is met, 
% of zelfs beter dan, de menselijke waarneming.
% \newline
% \newline
% Een van de meest fundamentele taken binnen computer vision is beeldclassificatie, waarbij een algoritme een afbeelding analyseert en deze toewijst aan 
% een vooraf gedefinieerde klasse, zoals "kat", "auto", of "boom". Hoewel dit op het eerste gezicht een eenvoudige taak lijkt, vormde het jarenlang een 
% grote uitdaging voor traditionele machine learning-methoden, die afhankelijk waren van handmatig geëxtraheerde kenmerken en beperkt waren in 
% hun generaliseerbaarheid \autocite{Geron2022}.
% \newline
% \newline
% De doorbraak in dit domein kwam met de opkomst van deep learning, en in het bijzonder convolutionele neurale netwerken (CNN's). 
% CNN's zijn in staat om automatisch ruimtelijke hiërarchieën van kenmerken te leren, waardoor ze uitblinken in het verwerken van visuele patronen \autocite{LeCun_2015}. 
% Sinds de overwinning van AlexNet op de ImageNet-competitie in 2012 \autocite{krizhevsky2012imagenet}, 
% zijn CNN-gebaseerde modellen de dominante benadering geworden voor talrijke computer vision-toepassingen, waaronder objectdetectie, beeldsegmentatie, 
% gezichtsherkenning en actieherkenning in video.
% \newline
% \newline
% Binnen de context van dit onderzoek ligt de focus op \textit{semantische segmentatie}, een taak die verder gaat dan eenvoudige classificatie. 
% Hierbij wordt elk afzonderlijk pixel in een afbeelding geclassificeerd als behorend tot een specifieke semantische categorie. 
% Deze fijnmazige benadering is essentieel in domeinen waar ruimtelijke precisie en context cruciaal zijn, zoals in medische beeldanalyse, 
% autonome voertuigen en satellietbeelden voor stedelijke en ecologische monitoring \autocite{zhu2017deep}.
% \newline
% \newline
% De recente vooruitgang in computer vision wordt mede mogelijk gemaakt door beschikbaarheid van grootschalige datasets (zoals ImageNet en COCO), 
% verhoogde rekenkracht (GPU's en TPU's), en nieuwe architecturen zoals Vision Transformers (ViTs), 
% die gebruik maken van self-attention mechanismen om contextuele informatie effectiever te modelleren \autocite{dosovitskiy2020image}.
% \newline
% \newline

% Semantic Segmentation
% \section{\IfLanguageName{dutch}{Semantische Segmentatie}{Semantic Segmentation}}%
% \label{sec:semantische-segmentatie}

% Semantische segmentatie is een computer vision taak dat elke pixel in een bepaalde klasse classificeerd aan de hand van machine learning 
% modellen \autocite{ibm2025a}. De output vormt een segmentatie kaart. Dit is een kopie van de originele foto waarbij elke pixel tot 
% een bepaalde segmentatie mask behoort. Masks zijn delen van de foto waarbij gelijkaardige pixels samen een object vormen \autocite{Geron2022}.
% Semantische segmentatie wordt bijvoorbeeld gebruikt voor AI sturende voertuigen en camera's om objecten zoals mensen te identificeren \autocite{ibm2025a}. 
% \newline
% \newline

% Semantische segmentatie is een fundamentele taak binnen het domein van computer vision waarbij elk afzonderlijk pixel in een afbeelding wordt geclassificeerd 
% als behorend tot een specifieke semantische klasse, zoals "weg", "vegetatie", "gebouw" of "lucht" \autocite{ibm2025a}. 
% In tegenstelling tot objectdetectie, dat enkel bounding boxes genereert rondom objecten, biedt semantische segmentatie een fijnmazige representatie 
% van objectgrenzen, wat essentieel is voor toepassingen waar precieze lokalisatie van klasse-informatie op pixelniveau vereist is.
% \newline
% \newline
% De output van een semantisch segmentatiemodel is een zogenaamde \textit{segmentatiekaart}, een afbeelding met dezelfde resolutie als de invoerafbeelding, 
% waarbij elke pixel een label draagt dat overeenkomt met de geclassificeerde klasse. Pixels die tot hetzelfde objecttype behoren, 
% vormen samen een \textit{mask}, of segment, wat het mogelijk maakt om objecten van dezelfde klasse consistent te identificeren, 
% ongeacht hun locatie in de afbeelding \autocite{garcia2020review}.
% \newline
% Moderne semantische segmentatiemodellen zijn doorgaans gebaseerd op convolutionele neurale netwerken (CNNs) en encoder-decoder architecturen. 
% De encoder (bijvoorbeeld ResNet of VGG) extraheert hiërarchische kenmerken uit het beeld door middel van convolutielagen, 
% terwijl de decoder deze abstracte representaties herprojecteert naar de originele resolutie met behulp van upsampling of transposed convoluties \autocite{ronneberger2015u}. 
% % Bekende modellen binnen dit domein zijn onder andere FCN (Fully Convolutional Network), U-Net, DeepLabv3+ en SegNet.
% \begin{figure}[h]
%   \caption{Voorbeeld van semantische segmentatie}
%   \centering 
%   \includegraphics{Sematische Segmentatie.png}
%   \end{figure}
% \newline
% \newline
% Een belangrijk aspect binnen semantische segmentatie is het behouden van ruimtelijke resolutie tijdens de opwaartse reconstructie van de afbeelding. 
% Technieken zoals skip connections (bijvoorbeeld in U-Net) worden gebruikt om fijnmazige kenmerken uit vroege lagen van het netwerk te combineren met 
% de globale contextinformatie van de diepere lagen, wat leidt tot een hogere nauwkeurigheid bij het detecteren van fijne objectranden \autocite{ronneberger2015u}.
% \newline
% In toepassingen zoals medische beeldvorming, autonome voertuigen en remote sensing, is semantische segmentatie een cruciaal hulpmiddel. 
% In remote sensing, bijvoorbeeld, maakt deze techniek het mogelijk om stedelijke infrastructuur, 
% vegetatiezones of waterlichamen op schaal te identificeren vanuit satellietbeelden met hoge resolutie, 
% wat van groot belang is voor milieumonitoring en stedelijke planning \autocite{zhu2017deep}.
% \newline
% \newline

% Machine Learning
\section{\IfLanguageName{dutch}{Machine Learning}{Machine Learning}}%
\label{sec:machine-learning}

Machine learning (ML) is een onderdeel van artificiële intelligentie (AI) gericht op het imiteren van menselijke leerprocessen, dat instaat 
is zelfstandig keuzes te maken aan de hand van training met gerichte data \autocite{ibm2025}. De term werd voor het eerst gebruikt in de 
jaren 50 door Arthur Samuel. Een pionier in de vroege ontwikkeling van AI en ML, meest bekend voor zijn computer model dat in staat 
was om de winkans op elk moment in een match dammen van beide spelers te bereken. 
\newline
\newline
Binnen machine learning heersen verschillende 'learning' technieken. De drie belangrijkste 
zijn supervised, unsupervised en reinforcement learning \autocite{Geron2022}. In dit onderzoek wordt gebruik gemaakt van supervised learning. Het model maakt 
gebruik van gelabelde data. Elke input heeft een gelabelde output. Dit kan manueel gedaan worden door mensen of automatisch aan de 
hand van een bestaand algoritme. Het idee achter gelabelde data is dat het ML model hieruit patronen kan vinden gebaseerd op de input output
relatie. 
\newline
\newline
Een Machine learning model wordt opgedeeld in verschillende stappen. De meest voorkomende zijn preprocessing, training en evaluatie.
Samen vormen ze een ML pipeline. Het preprocessing deel bevat operaties die op de data moeten worden uitgevoerd voordat de training van 
start kan gaan. Veel voorkomende technieken zijn data normalisatie, data herschalen enz \ldots. De trainingsfase behoud zich tot het
trainen van de data. Het model krijgt input (data) binnen en moet op basis van deze input een output geven. Bij supervised learning bevat
de input ook al de gewenste output (labels). De labeling kan handmatig worden uitgevoerd door menselijke annotators of 
automatisch worden gegenereerd door een bestaand algoritme.
\newline
\newline
Het doel van ML training is om het model te leren zelfstandig beslissingen te maken op basis van de input data. 
Wanneer een model voldoende getraind is, wordt het geëvalueerd met nieuwe, ongeziene data. De evaluatie is
vergelijkbaar met de trainingsfase met als verschil dat er nu geen labels worden meegegeven. Het model moet nu zelf de output voorspellen.
Tijdens de evaluatie worden een aantal beoordeling criteria gebruikt. De accuraatheid van het model (procent van output die overeenkomt met
label) bijvoorbeeld. Na de evaluatie wordt het model hertraind als de statistieken niet hoog genoeg zijn. Wanneer de evaluatie criteria behaald 
zijn kan het model gebruikt worden in een productie omgeving.
\newline
\newline

% Machine learning (ML) is een subdiscipline binnen artificiële intelligentie (AI) die zich richt op het nabootsen van menselijke 
% leerprocessen en in staat is om zelfstandig beslissingen te nemen op basis van training met gerichte datasets \autocite{ibm2025}. 
% De term `machine learning' werd voor het eerst geïntroduceerd in de jaren 1950 door Arthur Samuel, een pionier op het gebied van AI en ML, 
% die vooral bekend werd door zijn schaakprogramma dat de winstkansen van beide spelers kon berekenen. Er zijn twee soorten machine learning
% taken: classificatie en regressie. Bij classificatie behoord de output van het model tot een gedefineerde klasse. Bij regressie is de 
% output meer variabel. Meestal een float of integer waarde.
% \newline
% \newline
% Binnen machine learning bestaan ook verschillende leermethoden, waarvan de drie belangrijkste supervised-, unsupervised- en 
% reinforcement learning zijn. In dit onderzoek wordt gebruikgemaakt van supervised learning, een methode waarbij het model wordt getraind 
% met gelabelde data \autocite{mahesh2020machine}. Dit betekent dat elke input in de trainingsset correspondeert met een bekende, vooraf bepaalde output. 
% De labeling kan handmatig worden uitgevoerd door menselijke annotators of automatisch worden gegenereerd door een bestaand algoritme. 
% Het doel van gelabelde data is het identificeren van patronen in de input-output relatie, zodat het model zelfstandig voorspellingen kan 
% doen op basis van nieuwe gegevens.
% \newline
% \newline
% Een machine learning-model wordt doorgaans ontwikkeld in verschillende fasen, die samen een ML-pijplijn vormen. 
% De meest voorkomende stappen zijn preprocessing, training en evaluatie. De preprocessingfase omvat verschillende operaties die 
% noodzakelijk zijn om de ruwe data geschikt te maken voor training \autocite{Geron2022}. Veelgebruikte technieken in deze fase zijn data-normalisatie en rescaling.
% Tijdens de trainingsfase wordt het model getraind door middel van inputdata, waarbij het leert om de bijbehorende output te voorspellen. 
% In het geval van supervised learning betekent dit dat zowel de invoer als de gewenste uitvoer (labels) beschikbaar zijn. 
% Het doel van deze fase is het optimaliseren van het model zodat het zelfstandig beslissingen kan nemen op basis van nieuwe invoergegevens.
% Na de trainingsfase wordt het model geëvalueerd met nieuwe, ongeziene data. In deze evaluatiefase ontvangt het model enkel input, 
% zonder bijbehorende labels, en moet het zelfstandig een output genereren. De prestaties van het model worden beoordeeld aan de hand 
% van verschillende maatstaven, zoals nauwkeurigheid (het percentage correcte voorspellingen ten opzichte van de labels). 
% Indien de prestaties onvoldoende zijn, wordt het model opnieuw getraind en geoptimaliseerd totdat de evaluatiecriteria zijn behaald. 
% Zodra het model voldoet aan de gestelde eisen, kan het worden ingezet in een productieomgeving.
% \newline
% \newline

% Deep learning & Neurale netwerken
\section{\IfLanguageName{dutch}{Deep Learning en Neurale Netwerken}{Deep learning and Neural Networks}}%
\label{sec:deep-learning-neurale-netwerken}

Deep learning is een sub discipline van machine learning dat zich focust op artificiële neurale netwerken (ANN) \autocite{Geron2022}. 
ANN is, zoals de naam aangeeft, een model gebaseerd op de neurale netwerken van het menselijke brein. In 1943 werd in het onderzoek van 
\textcite{McCulloch1943} een artificiële neuron voorgesteld. Gelijkaardig aan biologische neuronen worden artificiële neuronen 
geactiveerd door (binaire) inputs. Dit maakt ze instaat om logische berekeningen op te lossen zoals het AND en OR logische probleem. 
\newline
\newline
De volgende doorbraak kwam in 1958 met het onderzoek van \textcite{Rosenblatt1958} met de preceptron. 
Rosenblatt maakte een nieuwe versie van de artificiële neuron genaamd threshold logic unit (TLU). 
De neuron heeft een set van weights op al de inputs plus één bias term.
Deze berekeningen worden gevolgd door twee activatie functies. Eerst een lineaire functie op alle inputs: \( w^t x + b \).
Dan volgt een step functie met formule: \(h_w(x) = step(z)\) \autocite{Geron2022}. De preceptron is opgebouwd uit meerdere TLU's en is in staat om binaire 
classificatie met lineaire data uit te voeren. In 1969 werd in het onderzoek van \textcite{Minsky1969} echter aangetoond dat de 
preceptron niet in staat was om het XOR probleem op te lossen. Hierdoor werd de multi layer perceptron (MLP) uitgevonden. Door meerdere 
lagen van TLU's aan elkaar te koppelen is het netwerk wel in staat om het XOR probleem op te lossen. De eerste laag TLU's wordt de input layer genoemd.
Dan volgt een bepaald aantal lagen die als de hidden layers worden aangeduid. Meer lagen betekend meer parameters en maakt het netwerk complexer.
De laatste laag, de output layer, bepaald de output van het netwerk. In het geval van RSCD binaire classificatie bestaat de ouput layer meestal 
uit één neuron die de waarde 0 of 1 weergeeft (0 = geen verandering, 1 = verandering).
\newline
\newline
Het backpropagation algoritme, voorgesteld in het onderzoek van \textcite{Rumelhart1986}, maakte neurale netwerken veel efficiënter.
Door twee keer door het netwerk te gaan, kan het algoritme bereken wat de invloed is van de weights en bias van elke neuron op het resultaat.
Tijdens de eerste stap, genaamd de foward pass, word de input op normale wijze door het netwerk gestuurd. Dan volgt de backward pass, waar
het algoritme de gradient error berekent van elke neuron. Het past dan de parameters van het netwerk aan om de error te minimaliseren. 
Dit proces blijft zich herhalen tot de error een minimum bereikt.
\newline
\newline
Tegenwoordig bestaan er veel soorten neurale netwerken met verschillende architecturen. In dit onderzoek wordt gefocust op 
convolutionele neurale netwerken, siamese netwerken en vision transformers.
\newline
\newline

% Deep learning is een subdiscipline binnen machine learning die zich richt op artificiële neurale netwerken (ANN) \autocite{LeCun_2015}. 
% Dit type algoritme is geïnspireerd op de werking van het menselijk brein en bestaat uit meerdere lagen van met elkaar verbonden neuronen. 
% Een neuraal netwerk is opgebouwd uit verschillende lagen, waarbij elke laag een groot aantal artificiële neuronen bevat die onderling 
% verbonden zijn met de neuronen in de volgende laag.
% \newline
% Elk neuron binnen het netwerk beschikt over een set parameters, namelijk weights (gewichten) en biases (bias-termen) \autocite{Geron2022}. 
% De weights bepalen de sterkte van het signaal tussen twee neuronen; een hogere waarde resulteert in een sterkere activatie van het neuron. 
% De bias is een parameter die bijdraagt aan de beslissing of een neuron geactiveerd wordt.
% \newline
% Neurale netwerken bestaan doorgaans uit drie hoofdcomponenten: de input layer, de hidden layers en de output layer. 
% De input layer ontvangt de invoergegevens en geeft deze door aan het netwerk. Vervolgens worden de gegevens verwerkt door een reeks 
% hidden layers, waarvan het aantal afhankelijk is van de complexiteit van de taak. De verwerking eindigt bij de output layer, 
% die de uiteindelijke voorspelling of classificatie genereert.
% \newline
% \newline
% Elke laag in het netwerk voert berekeningen uit op basis van de input uit de voorgaande laag en geeft de output door aan de volgende laag. 
% Hierdoor ontstaat een diep gelaagd netwerk, wat de term deep learning verklaart. Zowel de hidden layers als de output layer maken 
% gebruik van een activatiefunctie, die wordt toegepast op de output van de neuronen om niet-lineariteit in het model te introduceren en 
% de leerprocessen te verbeteren. Nadat de input het hele netwerk heeft doorkruisd wordt een belangrijk algoritme gebruikt om het model 
% te optimaliseren. Backpropagation (backward propagation of errors) gebruikt de error gradient om te berekenen hoe elke neuron's 
% output heeft geleid tot de predictie. Met deze gradient worden de weights dan optimaal aangepast. Vervolgens wordt de input weer door 
% het netwerk gestuurd. Dit noemen we de foward pass en Backpropagation wordt de backward pass genoemd. Deze twee worden herhaald tot de
% error van de output minimaal is.
% \newline
% \newline


% % Machine Learning Models
% \section{\IfLanguageName{dutch}{Deep Learning Modellen}{Deep Learning Models}}%
% \label{sec:machine-learning-models}

% Zoals andere computer visions taken heeft de opkomst van deep learning de meer tradionele methoden (Support Vector Machines and 
% Random Forest) grotendeels vervangen. Tijdens dit onderzoek zal ik de twee groepen met elkaar vergelijken.
% \newline
% \newline

% CNN
\section{\IfLanguageName{dutch}{Convolutionele Neurale Netwerken}{Convolutional Neural Networks}}%
\label{sec:convolutionele-neurale-netwerken}

Convolutional neurale netwerken (CNN) worden gezien als één van de beste machine learning algoritmes als het aankomt op computer vision \autocite{Geron2022}.
CNNs zijn neurale netwerken, geïntroduceerd door het onderzoek van \textcite{LeCun2002}, voor het herkennen van handschriften (MNIST).
Ze zijn opgebouwd uit convolutionele lagen. De neuronen in de input layer zijn alleen geconnecteerd met pixels in de 
overeenkomende receptive fields \autocite{LeCun2002}. De neuronen van de input layer zijn geconnecteerd met een klein aantal neuronen 
van de tweede laag enz \ldots. Deze architectuur maakt het ideaal om hiërarchische features en patronen te ontdekken in de input data \autocite{Bengio2017}.
\newline
\newline
De weights van de neuronen worden voorgesteld als filters. Ze worden gebruikt om de belangrijkste delen van de data te vinden. 
The output van een filter operatie is een feature map. De convolutionele laag zoekt automatisch voor de beste filters voor de data. 
Na een convolutionele laag volgt een activatie functie (ReLU functie) \autocite{krizhevsky2012imagenet}. Een convolutioneel netwerk bevat typische ook een aantal 
pooling layers. Pooling layers verminderen de computational load door de input te aggregeren. De meest gebruikte is de max pooling layer dat de 
max waarde van de filter doorgeeft. Op het einde van de CNN bevinden zich een aantal dense layers (TLU's) \autocite{Geron2022}.
Bij een binaire classificatie taak heeft de laatste laag maar één neuron. Als activatie functie gebruiken we de sigmoïd functie: \(\sigma(x) = \frac{1}{(1 + e^-x)} \)
% Dit houd in dat het netwerk voor elke klasse een percentage output als kans dat de afbeelding tot die klasse behoord. 
\newline
\newline
Onderzoeken zoals \textcite{zhu2017deep} en \textcite{Bai_2022} tonen aan dat CNN's geschikt zijn voor remote sensing change detection.
Omdat CNN in computer vision taken zoals object detection en semantische segmentatie goed scoorde, werden ze ook toegepast op change detection.
CNN's blijken ook goed in geospatiale metadata te verwerken.
\newline
De bekendste CNN architecturen zijn LeNet-5 \textcite{LeCun2002}, AlexNet \textcite{krizhevsky2012imagenet}, GoogLeNet \textcite{Szegedy2015}
en ResNet \textcite{He2016}. 
\newline
\newline

% Convolutionele neurale netwerken (CNN's) worden beschouwd als een van de meest effectieve machine learning-algoritmen voor visuele 
% verwerkingstaken. CNN's zijn opgebouwd uit meerdere convolutionele lagen, waarbij de neuronen in de input layer uitsluitend verbonden 
% zijn met pixels binnen hun respectieve receptive fields \autocite{Geron2022}. Op vergelijkbare wijze zijn de neuronen in elke laag 
% slechts gekoppeld aan een beperkt aantal neuronen in de daaropvolgende laag. Deze architectuur maakt CNN's bijzonder geschikt voor het 
% detecteren van hiërarchische kenmerken en patronen binnen de invoergegevens.
% \newline
% \newline
% De weights van de neuronen in een convolutionele laag worden voorgesteld als filters, die dienen om specifieke kenmerken in de 
% invoerdata te versterken. De uitvoer van een dergelijke filterbewerking resulteert in een feature map, waarin de belangrijkste patronen 
% en structuren van de oorspronkelijke invoer worden vastgelegd. De convolutionele laag optimaliseert deze filters automatisch om de meest 
% relevante kenmerken uit de gegevens te extraheren. Direct na een convolutionele laag wordt doorgaans een activatiefunctie toegepast, 
% meestal de ReLU-functie, om niet-lineariteit in het model te introduceren en de leercapaciteit te verbeteren.
% \newline
% \newline
% Naast convolutionele lagen bevatten CNN's vaak pooling layers, die de dimensie van de invoer reduceren en daarmee de rekenlast van 
% het model verlagen. De meest gebruikte techniek is max pooling, waarbij de maximale waarde binnen een bepaald filtergebied wordt 
% geselecteerd. Dit helpt bij het behouden van de meest informatieve kenmerken en verhoogt de robuustheid van het model.
% \newline
% \newline
% Aan het einde van het CNN bevinden zich doorgaans meerdere dense layers, die verantwoordelijk zijn voor de uiteindelijke classificatie 
% of voorspelling. Bij classificatieproblemen bevat de laatste laag een aantal neuronen dat overeenkomt met het aantal mogelijke klassen. 
% Hierbij wordt de softmax-activatiefunctie toegepast, wat resulteert in een waarschijnlijkheidsverdeling over de klassen. 
% Dit betekent dat het model voor elke klasse een waarschijnlijkheid berekent, die aangeeft in welke mate de invoer tot die klasse behoort.
% \newline
% \newline

% Siamese networks
\section{\IfLanguageName{dutch}{Siamese Netwerken}{Siamese Networks}}%
\label{sec:siamese-netwerken}

Siamese networks zijn voor het eerst voorgesteld in het onderzoek van \textcite{NIPS1993_288cc0ff}. Het is een algoritme dat bestaat uit
twee of meerdere identieke neurale netwerken met dezelfde weights en biases \autocite{Serrano_2023}.
In dit onderzoek zullen we werken met een twin network. Beide netwerken verwerken een deel van de dataset ((t₁)  (t₂)).
De outputs worden dan vergeleken met elkaar aan de hand van de euclidean distance loss functie. 
Een van de grootste voordelen ligt in het feit dat het in staat is accurate predicties te maken met weinig input data \autocite{koch2015siamese}. 
Het wordt daarom ook wel het one-shot model genoemd.
\newline
bevindingen uit het onderzoek van \textcite{Fang2019} tonen aan dat een twin siamese netwerk goed scoort met een relatief simpel model. 
Nadelen zijn de relatief trage trainingstijd en dat de convergentiecurven van het model ongebalanceerd zijn.
\newline
\newline

% Siamese netwerken zijn een klasse van neurale netwerkarchitecturen die oorspronkelijk werden geïntroduceerd door \textcite{NIPS1993_288cc0ff} voor de taak 
% van handschriftherkenning. De kern van een Siamese netwerk bestaat uit twee (of meer) identieke subnetwerken die parallelle inputs verwerken en waarvan de 
% parameters – waaronder gewichten en biases – volledig gedeeld worden \autocite{Serrano_2023}. Hierdoor wordt gegarandeerd dat beide subnetwerken 
% identieke transformaties uitvoeren op verschillende invoerdata.
% \newline
% De primaire functie van een Siamese netwerk is het leren van een vergelijkingsfunctie die de gelijkenis tussen twee invoerinstanties kan kwantificeren. 
% De uitgangen van de subnetwerken worden geprojecteerd naar een embeddingruimte waarin vergelijkbare paren dichter bij elkaar liggen dan niet-vergelijkbare paren. 
% Dit gebeurt meestal met behulp van een afgeleide afstandsmaat, zoals de Euclidische afstand, cosine-similariteit of de contrastieve loss-functie \autocite{hadsell2006dimensionality}. 
% Dit maakt het model bijzonder geschikt voor taken waarbij het onderscheid tussen gelijken en niet-gelijken cruciaal is.
% \newline
% In de context van dit onderzoek wordt gebruikgemaakt van een \emph{twin network} opstelling, 
% waarin twee identieke CNN-gebaseerde subnetwerken afzonderlijke temporele snapshots van dezelfde geografische locatie analyseren – bijvoorbeeld satellietbeelden 
% uit 2020 en 2024. De overeenkomst of verandering tussen deze beelden wordt geëvalueerd via een distance-metric loss, 
% waarmee veranderingen zoals stedelijke uitbreiding of vegetatieverlies gedetecteerd kunnen worden.
% \newline
% \newline
% Een belangrijk voordeel van Siamese netwerken is hun efficiëntie bij het leren van onderscheidende representaties op basis van een beperkt aantal 
% trainingsvoorbeelden. Hierdoor worden zij vaak geclassificeerd als \emph{one-shot} of \emph{few-shot} learning modellen \autocite{koch2015siamese}. 
% In tegenstelling tot traditionele classificatiemodellen die veel voorbeelden per klasse vereisen, kunnen Siamese netwerken nieuwe klassen generaliseren door 
% simpelweg gelijkenis te meten met eerder geziene voorbeelden. Dit maakt ze bijzonder waardevol in situaties met beperkte gelabelde data, 
% zoals zeldzame objectdetectie of adaptieve change detection in remote sensing.
% \newline
% Recent onderzoek heeft aangetoond dat Siamese netwerken met CNN-backbones effectief zijn in verschillende domeinen zoals gezichtsherkenning \autocite{chopra2005learning}, 
% medische beeldanalyse \autocite{zhang2021survey}, en veranderingdetectie in aardobservatiebeelden \autocite{shafique2022deep}. 
% Hun vermogen om directe vergelijkingen te maken tussen beeldparen, zonder dat volledige klassenstructuren vereist zijn, 
% maakt ze bij uitstek geschikt voor temporele beeldanalyse in dynamische stedelijke omgevingen.
% \newline
% \newline
% Samenvattend vormen Siamese netwerken een krachtige architectuur voor visuele vergelijkingstaken. Dankzij hun parameterdeling, efficiëntie bij weinig data, 
% en robuustheid bij one-shot learning, bieden zij een waardevolle benadering voor automatische change detection op basis van multispectrale 
% of optische satellietbeelden.
% \newline
% \newline

% Vision Transformer
\section{\IfLanguageName{dutch}{Vision Transformer}{Vision Transformer}}%
\label{sec:vision-transformer}

Transformers zijn neurale netwerken die uitsluitend bestaan uit attention layers voorgesteld in het onderzoek van \textcite{Vaswani2017}.
Attention is een methode om alleen relevante input data te verwerken in plaats van de hele batch \autocite{Geron2022}. 
Dit vermindert de trainingstijd zonder de nauwkeurigheid van het model significant te beïnvloeden. 
\newline
\newline
Vision transformer (ViT) is een transformer gebouwd voor visuele taken. Oorspronkelijk geïntroduceerd in het onderzoek van \textcite{dosovitskiy2020image}.
In deze studie bouwde onderzoekers een encoder only trasformer. Het resultaat was een state-of-the-art model dat beter scoorde dan CNN tegenhangers.
ViT's zijn, dankzij de transformer-architectuur, bijzonder goed afgestemd op grootschalige datasets, wat bijdraagt aan hun huidige populariteit
tijdens de opmars van big data. 
De input wordt gesplist in gelijke stukjes (bijvoorbeeld 16\times16 pixels) genoemd embeddings \autocite{dosovitskiy2020image}. Dan volgt 
een transformer encoder die bestaat uit meerdere multi head self attention lagen. Tot slot een MLP gevolgd door een softmax laag 
voor classificatie.
\newline
\newline
Een belangrijk voordeel van ViTs ten opzichte van CNNs is hun capaciteit om globale contexten vroeg in het model te leren, zonder gefixeerde convolutiekernen. 
Hierdoor zijn ViTs zeer effectief op grootschalige datasets zoals ImageNet21k of JFT-300M \autocite{dosovitskiy2020image}. Desondanks tonen empirische studies aan 
dat ViTs significant meer trainingsdata nodig hebben dan CNNs om vergelijkbare generalisatieprestaties te bereiken, vooral in low-data regimes \autocite{touvron2021training}. 
Dit heeft geleid tot de ontwikkeling van hybride modellen zoals DeiT (Data-efficient Image Transformer) \autocite{touvron2021training}, 
die trainingsefficiëntie verhogen via technieken zoals distillation tokens.
\newline
\newline
Recent onderzoek zoals \textcite{Bandara2022} toont aan dat vision transformers beter kunnen scoren op remote sensing change detection
dan de meer traditionele CNN modellen.
\newline
\newline


% Transformers zijn neurale netwerkarchitecturen die uitsluitend gebruikmaken van attention-mechanismen en voor het eerst werden geïntroduceerd in het 
% baanbrekende onderzoek van \textcite{Vaswani2017}. In tegenstelling tot conventionele sequentiële modellen zoals recurrent neural networks (RNNs), 
% verwerken Transformers alle invoer tegelijkertijd en benutten zij het mechanisme van self-attention om dynamisch te bepalen welke delen van de inputreeks 
% relevant zijn voor een bepaalde voorspelling. Dit leidt tot een hoge mate van paralleliseerbaarheid tijdens training, terwijl het model toch complexe, 
% lange-afstand relaties kan modelleren \autocite{Geron2022}.
% \newline
% \newline
% Een innovatieve uitbreiding van deze architectuur naar het domein van computer vision is de \emph{Vision Transformer} (ViT), geïntroduceerd door 
% \textcite{dosovitskiy2020image}. In tegenstelling tot conventionele convolutionele neurale netwerken (CNNs), die gebaseerd zijn op lokale receptieve 
% velden en translationele invariantie, behandelt ViT een beeld als een reeks gestructureerde sequenties van beeldpatches, 
% wat een fundamentele verschuiving in paradigma vertegenwoordigt.
% \newline
% \newline
% Het standaardinvoerbeeld $x \in \mathbb{R}^{H \times W \times C}$ wordt eerst opgedeeld in $N$ niet-overlappende patches van vaste grootte, 
% bijvoorbeeld $16 \times 16$ pixels, waarna elk patch wordt geprojecteerd naar een d-dimensionale vector via een lineaire embedding. 
% Deze geprojecteerde vectoren worden vervolgens verrijkt met \emph{positionele encoderingen} om ruimtelijke informatie te behouden, 
% aangezien de self-attention-mechanismen zelf positionele structuur negeren.
% \newline
% De resulterende sequentie van tokens wordt ingevoerd in een standaard Transformer encoder bestaande uit gestapelde lagen van multi-head self-attention (MSA) 
% en feedforward netwerken (MLP), elk gevolgd door layer normalization en residual connections. 
% % Een extra geclassificeerd token (*[CLS]*) 
% % wordt toegevoegd aan het begin van de sequentie. De uiteindelijke representatie van dit token wordt geïnterpreteerd door een MLP-head voor classificatie.
% \newline
% \newline
% Een belangrijk voordeel van ViTs ten opzichte van CNNs is hun capaciteit om globale contexten vroeg in het model te leren, zonder gefixeerde convolutiekernen. 
% Hierdoor zijn ViTs zeer effectief op grootschalige datasets zoals ImageNet21k of JFT-300M \autocite{dosovitskiy2020image}. Desondanks tonen empirische studies aan 
% dat ViTs significant meer trainingsdata nodig hebben dan CNNs om vergelijkbare generalisatieprestaties te bereiken, vooral in low-data regimes \autocite{touvron2021training}. 
% Dit heeft geleid tot de ontwikkeling van hybride modellen zoals DeiT (Data-efficient Image Transformer) \autocite{touvron2021training}, 
% die trainingsefficiëntie verhogen via technieken zoals distillation tokens.
% \newline
% \newline
% In remote sensing en change detection-taken worden Vision Transformers steeds vaker toegepast vanwege hun vermogen om lange-afstandspatronen en 
% contextuele relaties in hoge-resolutiebeelden te modelleren \autocite{xu2022vision}. Vooral bij stedelijke vegetatieanalyse, 
% waarbij heterogeniteit en fijne structuurkenmerken essentieel zijn, tonen ViTs potentieel superieure prestaties dankzij hun ruimtelijke flexibiliteit.
% \newline
% \newline
% Samenvattend bieden Vision Transformers een krachtige, schaalbare benadering voor beeldverwerkingstaken die complementair is aan CNN-gebaseerde methoden. 
% Hun vermogen om rijke contextuele relaties te modelleren, gecombineerd met architecturale eenvoud en paralleliseerbaarheid, 
% maken ze tot een belangrijk onderzoeksgebied binnen de hedendaagse computer vision.
% \newline
% \newline

% % Non Deep Learning modellen
% \section{\IfLanguageName{dutch}{Traditionele Machine Learning Models}{Traditional Machine Learning Models}}%
% \label{sec:traditionele-machine-learning-models}

% Buiten deep learning zijn er ook nog andere AI modellen die gebruikt worden voor CD.
% Deze studie focust op twee algoritmen: Support Vector Machines (SVM) en Random Forests.
% \newline
% \newline

% Random forests
\section{\IfLanguageName{dutch}{Random Forests}{Random Forests}}%
\label{sec:random-forests}

Random forests is een ensemble learning model dat bestaat uit meerdere decision trees algoritmen, 
waarbij elke tree gebaseerd is op willekeurige vectoren met dezelfde verdeling \autocite{Breiman_2001}. 
Ensemble learning is machine learning techniek waar gebruikt gemaakt wordt van meerdere modellen die één voor één getraind worden op de data.
Het Random Forest algoritme is opgebouwd uit meerdere decision trees. Het decision tree algoritme is opgebouwd zoals een boom.
De data wordt beweegt door de boom en komt zo in aanmerking met een decision tak.
\newline 
\newline
Het basisidee achter Random Forests is het genereren van meerdere decision trees door middel van bootstrap aggregating (bagging), waarbij 
elke boom wordt getraind op een willekeurige subset uit de trainingsdata. Bovendien wordt bij elke splitsing van de boom slechts een 
willekeurige subset van de features overwogen. Deze dubbele randomisatie, zowel op het niveau van observaties als van kenmerken, vermindert de correlatie 
tussen individuele bomen en verhoogt daarmee de algehele accuraatheid van het ensemble model \autocite{liaw2002classification}.
\newline 
\newline
Random Forests zijn bij uitstek geschikt voor het modelleren van complexe, niet-lineaire relaties in data zonder veel parameterafstemming. 
Ze zijn relatief ongevoelig voor outliers en multicollineariteit, en bieden daarnaast een ingebouwde methode voor het schatten van feature importance, 
wat de interpretatie van modellen in wetenschappelijke toepassingen vergemakkelijkt \autocite{pal2005random}.
Meerdere studies zoals \textcite{Wessels_2016} en \textcite{Feng_2018} tonen aan dat 
random Forest geschikt is voor change detection.
\newline
\newline

% Random Forests is een krachtig ensemble learning-algoritme dat gebruik maakt van een verzameling decision trees om zowel classificatie- als regressietaken 
% uit te voeren \autocite{Breiman_2001}. In tegenstelling tot een enkelvoudige decision tree, die gevoelig kan zijn voor overfitting, biedt een Random Forest 
% aanzienlijke robuustheid en generaliseerbaarheid door het combineren van meerdere, relatief zwakke leerders tot één sterk model.
% \newline
% Het basisidee achter Random Forests is het genereren van meerdere decision trees door middel van bootstrap aggregating (ook wel bagging genoemd), waarbij 
% elke boom wordt getraind op een willekeurige steekproef met teruglegging uit de trainingsdata. Bovendien wordt bij elke splitsing van de boom slechts een 
% willekeurige subset van de features overwogen. Deze dubbele randomisatie, zowel op het niveau van observaties als van kenmerken, vermindert de correlatie 
% tussen individuele bomen en verhoogt daarmee de algehele accuraatheid van het ensemble model \autocite{liaw2002classification}.
% \newline
% \newline
% Een decision tree binnen dit ensemble volgt een hiërarchische structuur waarbij interne knooppunten beslissingen nemen op basis van drempelwaarden in specifieke 
% kenmerken (bijv. spectrale waarden van pixels in remote sensing). Het blad van een boom vertegenwoordigt een eindclassificatie of waarde. 
% Elke afzonderlijke boom in het bos levert een voorspelling, en in het geval van classificatie wordt de uiteindelijke voorspelling bepaald door een meerderheid 
% van stemmen (majority voting). Voor regressie wordt vaak het gemiddelde van alle voorspellingen genomen.
% \newline
% Random Forests zijn bij uitstek geschikt voor het modelleren van complexe, niet-lineaire relaties in data zonder veel parameterafstemming. 
% Ze zijn relatief ongevoelig voor outliers en multicollineariteit, en bieden daarnaast een ingebouwde methode voor het schatten van feature importance, 
% wat de interpretatie van modellen in wetenschappelijke toepassingen vergemakkelijkt \autocite{pal2005random}.
% \newline
% \newline
% In de context van remote sensing, en specifiek bij change detection, zijn Random Forests bijzonder effectief gebleken. Studies zoals \textcite{Wessels_2016} en 
% \textcite{Feng_2018} benadrukken de prestaties van Random Forests bij het detecteren van veranderingen in tijdreeksen van satellietbeelden. Dit succes is 
% grotendeels te danken aan hun vermogen om heterogene landschapskenmerken te modelleren en subtiele verschillen in spectrale responsen te identificeren. 
% Bovendien presteren Random Forests goed in situaties met beperkte gelabelde data, wat van groot belang is in operationele omgevingen waar grondwaarheidsgegevens 
% schaars zijn.
% \newline
% Samenvattend biedt het Random Forest-algoritme een krachtige, interpreteerbare en robuuste benadering voor classificatieproblemen in remote sensing. 
% Door het combineren van meerdere bomen minimaliseert het model de kans op overfitting en maximaliseert het de nauwkeurigheid, waardoor het een waardevolle 
% tool is voor het uitvoeren van stedelijke vegetatie-analyse en veranderingen door de tijd.
% \newline
% \newline

% Support vector Machines
\section{\IfLanguageName{dutch}{Support Vector Machines}{Support Vector Machines}}%
\label{sec:support-vector-machines}

Support vector machine (SVM) is een supervised learning model dat gebruikt word voor beide classificatie en regressie \autocite{Hearst1998}.
Het model berekent aan de hand van decision boundaries, een hyperlane die de verschillende klassen opdeelt \autocite{Hearst1998}. 
In gevallen waarin de gegevens lineair scheidbaar zijn, functioneert SVM bijzonder efficiënt en biedt het vaak betere generalisatie dan 
andere lineaire modellen. 
\newline
\newline
Echter, veel real-world datasets, met name die afkomstig uit remote sensing toepassingen, zijn zelden lineair scheidbaar. 
Om deze beperking te overwinnen, introduceert SVM een methode om de inputruimte te transformeren naar een hogere dimensionale ruimte waarin 
de gegevens wél lineair te scheiden zijn. Dit proces wordt formeel gerealiseerd via een functie $\phi(\cdot)$ die de oorspronkelijke 
vectorruimte afbeeldt naar een hogere dimensionale Hilbertruimte. Echter zijn dit rekenkundig dure berekeningen.
\newline
\newline
Een oplossing hiervoor is het kernel trick algoritme, dat in staat is niet lineaire data van elkaar te onderscheiden zonder de data te 
transformeren naar een hogere dimensie. Deze techniek maakt het mogelijk om het inwendige product $\langle \phi(x_i), \phi(x_j) \rangle$ 
te berekenen zonder ooit expliciet $\phi(x)$ te hoeven bepalen. 
Veel gebruikte kernels zijn onder meer de Radial Basis Function (RBF), polynomiale en sigmoid kernels, die elk verschillende vormen van 
niet-lineariteit kunnen modelleren \autocite{cristianini2002support}.
\newline
Onderzoek, zoals dat van \textcite{Habib_2009} toont aan dat SVM's geschikt zijn voor change detection.

% Support Vector Machine (SVM) is een krachtig supervised learning model dat zijn oorsprong vindt in de statistische leerliteratuur en zich richt op zowel 
% classificatie- als regressietaken \autocite{mahesh2020machine}. Het fundamentele principe van SVM is het vinden van een optimale scheidingsgrens, 
% ofwel een hypervlak (hyperplane), die de grootste marge garandeert tussen de trainingsvoorbeelden van verschillende klassen. Deze marges worden 
% gedefinieerd door de dichtstbijzijnde gegevenspunten, bekend als *support vectors* \autocite{cortes1995support}.
% \newline
% \newline
% In gevallen waarin de gegevens lineair scheidbaar zijn, functioneert SVM bijzonder efficiënt en biedt het vaak betere generalisatie dan andere lineaire modellen. 
% Echter, veel real-world datasets, met name die afkomstig uit remote sensing toepassingen, zoals stedelijke vegetatieclassificatie, zijn zelden lineair scheidbaar. 
% Om deze beperking te overwinnen, introduceert SVM een methode om de inputruimte te transformeren naar een hogere dimensionale ruimte waarin de gegevens wél 
% lineair te scheiden zijn. Dit proces wordt formeel gerealiseerd via een functie $\phi(\cdot)$ die de oorspronkelijke vectorruimte afbeeldt naar een hogere 
% dimensionale Hilbertruimte.
% \newline
% \newline
% Hoewel directe transformatie naar deze hoge dimensies computationeel onhaalbaar kan zijn, biedt de zogeheten *kernel trick* een elegante oplossing. 
% Deze techniek maakt het mogelijk om het inwendige product $\langle \phi(x_i), \phi(x_j) \rangle$ te berekenen zonder ooit expliciet $\phi(x)$ te hoeven bepalen. 
% Veelgebruikte kernels zijn onder meer de Radial Basis Function (RBF), polynomiale en sigmoid kernels, die elk verschillende vormen van niet-lineariteit 
% kunnen modelleren \autocite{cristianini2002support}.
% \newline
% \newline
% Een bijzonder relevant toepassingsgebied van SVM binnen remote sensing is *change detection*, oftewel het opsporen van veranderingen in tijdreeksen van 
% geospatiale data. Uit studies, zoals die van \textcite{Habib_2009}, blijkt dat SVM's bijzonder effectief zijn in het detecteren van subtiele veranderingen in 
% complexe en heterogene omgevingen. Dit is grotendeels te danken aan hun vermogen om met beperkte trainingsdata toch een robuust beslissingsmodel te construeren, 
% wat essentieel is in situaties waar gelabelde data schaars of kostbaar is.
% \newline
% \newline
% Bovendien toont recent onderzoek aan dat, hoewel deep learning methoden zoals convolutionele neurale netwerken (CNNs) superieure prestaties kunnen leveren 
% bij grote datasets, traditionele modellen zoals SVM's nog steeds uitblinken in situaties met beperkte data of hoge ruisniveaus \autocite{zhu2017deep}. 
% In de context van stedelijke vegetatieanalyse biedt SVM een interpreteerbaar en minder data-intensief alternatief dat geschikt is voor operationele 
% implementaties met beperkte middelen.
% \newline
% \newline
% Door deze systematische vergelijking tussen deep learning en traditionele machine learning-modellen biedt dit onderzoek een grondige analyse van 
% welke benadering het meest geschikt is voor remote sensing change detection, met specifieke aandacht voor stedelijke vegetatie als casus.





% ML algoritmen zijn in staat om iteratief verbetering toe te brengen. Het 
% resultaat hangt af van verschillende factoren. De opbouw van het model, de data die het model gerbuikt. Ook hoe de data voorbereid word 
% (preprocessing) speelt een belangrijke rol.

%%=============================================================================
%% Methodologie
%%=============================================================================

\chapter{\IfLanguageName{dutch}{Methodologie}{Methodology}}%
\label{ch:methodologie}

%% TODO: In dit hoofstuk geef je een korte toelichting over hoe je te werk bent
%% gegaan. Verdeel je onderzoek in grote fasen, en licht in elke fase toe wat
%% de doelstelling was, welke deliverables daar uit gekomen zijn, en welke
%% onderzoeksmethoden je daarbij toegepast hebt. Verantwoord waarom je
%% op deze manier te werk gegaan bent.
%% 
%% Voorbeelden van zulke fasen zijn: literatuurstudie, opstellen van een
%% requirements-analyse, opstellen long-list (bij vergelijkende studie),
%% selectie van geschikte tools (bij vergelijkende studie, "short-list"),
%% opzetten testopstelling/PoC, uitvoeren testen en verzamelen
%% van resultaten, analyse van resultaten, ...
%%
%% !!!!! LET OP !!!!!
%%
%% Het is uitdrukkelijk NIET de bedoeling dat je het grootste deel van de corpus
%% van je bachelorproef in dit hoofstuk verwerkt! Dit hoofdstuk is eerder een
%% kort overzicht van je plan van aanpak.
%%
%% Maak voor elke fase (behalve het literatuuronderzoek) een NIEUW HOOFDSTUK aan
%% en geef het een gepaste titel.

% \lipsum[21-25]

Het doel van dit onderzoek is het ontwikkelen van een machine learning-pijplijn die effectief veranderingen 
kan detecteren tussen de aanwezigheid van vegetatie tussen twee tijdstippen. Dit is een vergelijkende studie, 
waarbij verschillende machine learning modellen getraind en geëvalueerd worden.
\newline
\section{\IfLanguageName{dutch}{Image preprocessing}{Image preprocessing}}%
\label{sec:image-preprocessing}
De eerste fase van het onderzoek is het samenstellen van twee datasets opgebouwd uit luchtfoto's van Geopunt Vlaanderen. De afbeelding hebben
het jpeg 2000 formaat. De training dataset bestaat uit afbeeldingen van verschillende Vlaamse steden tussen 2020 (t₁) en 2024 (t₂). 
Tijdens de trainingsfase wordt deze dataset gebruikt om het model te leren veranderingen in vegetatie te ontdekken tussen de twee tijdstippen.
De test dataset bestaat uit afbeeldingen van Gent tussen 2020 (t₁) in 2024 (t₂). Deze dataset wordt gebruikt voor model evaluatie.
\newline
Data wordt ingeladen met rasterio, een python library voor het gebruiken van GIS data. De afbeeldingen worden herschaalt naar (2560, 2560, 3).
Hierna wordt de foto opgedeeld in 100 (256, 256, 3) patches. De foto's worden gelabeld met behulp van de excess green index (ExG).
Deze index maakt gebruik van de rgb waarden van een afbeelding om vegetatie automatisch te laten generen: \(ExG = 2 * Green - Red - Blue \).
% het QUPath programma. In beide foto's wordt de vegetatie aangeduid. 
Hieruit ontstaan twee distincte vegetatie masks. Door de mask van (t₂) wiskundig te verminderen met die van (t₁) 
bereken we change labels tussen beide afbeeldingen. De training data wordt opgeslagen in een python lijst bestaande uit tuples van 
voor en na afbeeldingen (ndarray's). De lijst heeft een lengte van 600. 
% De test data (afbeeldingen van Gent) bestaat 100 tupels van afbeeldingen.
\newline
\section{\IfLanguageName{dutch}{CGNet}{CGNet}}%
\label{sec:cgnet}
Change guidding network (CGNet) is een convolutional neuraal netwerk voorgesteld in de studie van \textcite{Han2023}. 
Een model ontworpen voor change detection als reactie op de tekortkomingen van de UNet architectuur in verband met edge detection en 
internal holes.
\newline
\newline
\begin{figure*}
  \centering
  \includegraphics[width=\textwidth]{CGNet.png}
  \caption{\label{fig:CGNet}CGNet architectuur \autocite{Han2023}.}
\end{figure*}
Het model is opgedeeld in drie delen. Een encoder bestaande uit vijf VGG-16 blocks gevolgd door batch normalisatie. De decoder waar 
feature extractie wordt toegepast. Features van de encoder worden toegepast op een convolutional blocks opgemaakt uit een convolutional
layer, batch normalisatie en de ReLU activatie functie. De change map ontstaat door deep features, output van de eerste convolutional block
van de decoder, door een convolutional block gevolgd door een classificatie layer te sturen. De change guide module (CGM) is een self-attention 
architectuur voor het identificeren van de features die veranderingen tussen de twee afbeeldingen tonen. Deze features worden doorgeven, 
terwijl de andere worden gedropt. De output van de CGM's en de decoder worden samengevoegd en na twee convolutional blocks 
geclassificeerd in een change map als output van het model.
\newline
\newline
De pythorch library wordt gebruikt om de data klaar te maken voor het model. De dataset class transformeert de foto's van ndarray's naar 
pythorch tensors. De pythorch dataloader class biedt een itereerbare methode over de gegeven dataset. De batch size van de dataloader is vier en
de shuffle optie staat aan.
\newline
\newline
Tijdens de model training wordt de BCE With Logits Loss gebruikt. Deze loss functie 
combineert een sigmoïd functie en de BCELoss functie in één enkele klasse. Deze versie is numeriek stabieler dan een gewone Sigmoïd functie. 
De formule van de loss functie: \(l(x,y) = L = \{l_1,\ldots,l_N\}^T \), \(l_n = -w_n[y_n * \log\sigma(x_n) + (1-y_n) * 
\log(1 - \sigma(x_n))] \), waar N de batch size is.
\newline
\newline
Als optimizer gebruiken we adam. De Adaptive Moment Estimation combineert de voordelen van Momentum- en RMSprop-technieken om de 
leersnelheid tijdens de training aan te passen. Het werkt goed met grote datasets en complexe modellen omdat het geheugen efficiënt 
wordt gebruikt en de leersnelheid voor elke parameter automatisch wordt aangepast. Het model wordt getraind voor 10 epochs.

\section{\IfLanguageName{dutch}{ChangeFormer}{ChangeFormer}}%
\label{sec:change-former}
Changeformer is een siamese transformer neuraal netwerk voor change detection geïntroduceerd in het onderzoek van \textcite{Bandara2022}.
Het netwerk bestaat uit twee siamese twin transformer encoders, opgebouwd uit transformer en downsampeling blocks. Downsampeling van de input 
gebeurt initieel met een convolutional layer met Kernel=3, Stride=2 en Padding=1. De eerste downsampeling block heeft een convolutional layer 
met Kernel=7, Stride=4 en Padding=3. Na een downsampeling block volgt altijd een een transformer block. Dit is een 
self-attention layer met als doel de complexiteit van de input te verminderen door alleen de belangrijkste features te behouden. 
Er zijn vier difference modules die aan de hand van de features van de encoder veranderingen tussen de twee inputs berekenen. Het bestaat
uit batch normalisatie en een convolutionele layer gevolgd door de ReLU activatie functie. De weights van beide encoders worden met elkaar gedeeld.
\newline
\newline
\begin{figure*}
  \centering
  \includegraphics[width=\textwidth]{ChangeFormer.png}
  \caption{\label{fig:CF}ChangeFormer architectuur \autocite{Bandara2022}.}
\end{figure*}
De MLP decoder gebruikt de output van de finale difference module (vier feature difference maps) om zo de change map te voorspellen.
Eerst is er een MLP layer gevolgd door een upsampling layer die de feature difference maps herschalen naar de originele afmeting van de input. 
Hierna worden alle feature maps samengevoegd met behulp van een MLP layer. Met nog een upsampling layer en tenslotte een classificatie MLP layer. 
\newline
\newline
De data wordt ge preprocessed door de dataset en dataloader pythorch classes. De foto's worden omgezet in pythorch tensors. De loader heeft een
batch size = 4. Het model is gedeclareerd met de decoder softmax optie op false. Dit komt omdat we binaire en geen multi-label classificatie doen.
Als loss functie gebruiken we de Cross Entropy Loss. Dit criterium berekent het cross entropy loss tussen input logits en target values.
Het model gebruikt de adam optimizer. De data wordt getraind in 10 epochs.
\newline
\newline

\section{\IfLanguageName{dutch}{Random Forest}{Random Forest}}%
\label{sec:random-forest}
Voor dit deel van het onderzoek wordt de RandomForestClassifier class van de sklearn library gebruikt. Eerst reshapen we de training data 
naar een 2D array waarin elke rij een pixel voorstelt. Vervolgens wordt de data van (t₁) en (t₂) gecombineerd tot één feature. 
De change masks (labels) worden omgezet naar een 1D-array. Het model wordt geïnitialiseerd met 100 decision trees (n estimators). 
Dan start het trainingsproces. Elke paar afbeeldingen wordt apart getraind.
\newline
\newline

\section{\IfLanguageName{dutch}{SVM}{SVM}}%
\label{sec:svm}
Het Support vector machine model dat gebruikt wordt is afkomstig van de sklearn library. De features worden handmatig berekent. 
We reshapen de features naar een 2D-array waarbij elke pixel een rij is. De labels worden omgezet naar een 1D-array. Voor het model gebruiken we 
de rfb kernel en zetten probability op false. Dit doen we om het trainingsproces te versnellen. Het model wordt apart getraind op elk foto paar.
\newline
\newline

% Het doel van dit onderzoek is de ontwikkeling van een machine learning-pijplijn die op een efficiënte en nauwkeurige wijze veranderingen 
% in vegetatiebedekking kan detecteren tussen twee verschillende tijdstippen. Dit onderzoek omvat een vergelijkende studie waarbij 
% verschillende machine learning-modellen worden geëvalueerd op hun effectiviteit in change detection.
% \newline
% \newline
% De eerste stap in de methodologie betreft de constructie van twee datasets bestaande uit luchtfoto's genomen door het agentschap 
% digitaal Vlaanderen (middenschalige orthofotobedekking van het Vlaamse Gewest). De eerste dataset bevat afbeeldingen van de stad Gent 
% uit 2020 (t₁), terwijl de tweede dataset bestaat uit beelden van Gent uit 2024 (t₂). De analyse wordt uitgevoerd door de beelden 
% van t₁ en t₂ met behulp van machine learning-methoden te vergelijken om vegetatieveranderingen te identificeren.
% \newline
% \newline
% De datasets worden onderverdeeld in drie segmenten: training (60\%), validatie (20\%) en testing (20\%). 
% De trainingsfase dient om het model te optimaliseren op basis van gelabelde data, terwijl de validatiefase wordt gebruikt om de prestaties 
% van het model te evalueren op niet eerder geziene gegevens, waardoor overfitting wordt geminimaliseerd. 
% De testfase wordt vervolgens ingezet om de uiteindelijke modelprestaties te beoordelen. Een representatieve dataset is hierbij 
% essentieel om de generaliseerbaarheid van het model te waarborgen, zodat seizoensgebonden variaties en andere omgevingsfactoren correct 
% worden verwerkt.
% \newline
% \newline
% Vooraf gedefinieerde machine learning-modellen worden getraind en getest op de dataset, waarbij gebruik wordt gemaakt van 
% pretrained modellen in plaats van een model vanaf nul op te bouwen. De dataset wordt vooraf verwerkt via een preprocessing-stap, 
% waarbij technieken zoals rescaling (data-augmentatie) worden toegepast om alle afbeeldingen naar een uniforme resolutie te schalen. 
% De data wordt geannoteerd via semantische segmentatie, een computer vision-techniek waarbij elke pixel wordt toegewezen aan een specifieke 
% klasse (bijvoorbeeld vegetatie, gebouwen of wegen). Dit resulteert in een segmentatiekaart die een gedetailleerd overzicht biedt van de 
% objecten binnen de afbeelding.
% \newline
% \newline
% Om veranderingen tussen t₁ en t₂ te detecteren, worden verschillen in kenmerken zoals gemiddelde kleur, objectgrootte en vormvariatie 
% geanalyseerd. Vervolgens worden deze veranderingen geclassificeerd, waarbij niet alleen de aanwezigheid van verandering wordt vastgesteld, 
% maar ook de aard van de transformatie binnen de objecten.
% \newline
% \newline
% Voor de evaluatie van de verschillende machine learning-modellen worden meerdere prestatie-indicatoren gehanteerd. 
% De Vegetation Condition Index (VCI) wordt als primaire metriek gebruikt om veranderingen in vegetatiebedekking te kwantificeren door 
% de Normalized Difference Vegetation Index (NDVI)-waarden van t₁ en t₂ met elkaar te vergelijken. Daarnaast worden aanvullende 
% evaluatiemethoden toegepast, waaronder confusion matrix-gebaseerde metrieken zoals precision en recall, evenals Intersection over Union (IoU)
% om de nauwkeurigheid van de voorspelde veranderingen te beoordelen.
% \newline
% \newline
% De implementatie van het machine learning-proces wordt uitgevoerd met behulp van de twee belangrijkste Python-frameworks voor 
% machine learning: TensorFlow en Scikit-learn.

\chapter{\IfLanguageName{dutch}{Model Evaluatie}{Model Evaluation}}%
\label{ch:model-evaluatie}

Om de modellen te evalueren worden verschillende statistieken gebruikt. Deze methoden zijn geselecteerd op basis van gebruikte evaluatie 
technieken in recente studies rond RSCD zoals \textcite{Bandara2022} en \textcite{Han2023}. 

\section{\IfLanguageName{dutch}{Acuuraatheid score}{Accuracy score}}%
\label{sec:acuuraatheid-score}
Beginnende met de klassieke accuraatheid score. 
De output van het model wordt vergeleken met de overeenkomstige labels. Meer bepaald wordt het aantal juiste predicties gedeeld door het 
totaal aantal predicties: 
\newline
\begin{center} 
\( accuracy = \frac{Aantal \space juiste \space predicties}{Totaal \space aantal \space predicties}\) 
\end{center}.

\section{\IfLanguageName{dutch}{Precision score}{Precision score}}%
\label{sec:precision-score}
De precision score is een statistiek die meer inzicht bied op het vermogen van een classifier om negatieve data als positief te labelen.
In het geval van RSCD betekend dit de verhouding van pixels waar geen verandering is tussen tijdstip (t₁) en (t₂) die effectief als 0 
(geen verandering) gelabeld worden. Precision kan berekend worden door het aantal true positives te delen door het aantal true positives
en false positives: 
\newline
\begin{center} 
\(precision = \frac{tp}{(tp + fp)}\) 
\end{center}

\section{\IfLanguageName{dutch}{Recall score}{Recall score}}%
\label{sec:recall-score}
Recall is een evaluatie techniek waarbij het model getoetst wordt op basis van zijn vermogen om positieve data te identificeren. 
Het beoordeeld het aantal pixel waar verandering tussen beide tijdstip is waargenomen. Door de true positives te delen door de som van 
true positives en false negatives: 
\newline
\begin{center} 
\(recall = \frac{tp}{(tp + fn)}\) 
\end{center}

\section{\IfLanguageName{dutch}{Intersection over Union (IoU)}{Intersection over Union (IoU)}}%
\label{sec:iou-score}
De jaccard score of intersection over union (IoU) is een evaluatie criteria veel gebruikt binnen computer vision-taken. Het berekend
de overlap van de predictie en labels over de hele afbeelding. Voor RSCD is dit de voorspelde veranderingen over de gelabelde veranderingen.
Voor binaire classificatie problemen wordt IoU berekent door de true positives te delen door de som van true positives, false positves en false negatives.
\newline
\begin{center}
\(Jaccard \space index = \frac{tp}{tp + fp + fn} \)
\end{center}

\section{\IfLanguageName{dutch}{F1 score}{F1 score}}%
\label{sec:f1-score}
Als laatste statistiek testen we de f1 score. De F1-score kan worden geïnterpreteerd als een gemiddelde van de precision en
recall evaluatie methodes, waarbij een F1-score zijn beste waarde bereikt bij 1 en zijn slechtste score bij 0.
De formule voor de F1-score is: 
\newline
\begin{center}
\(f1 = \frac{2 * tp}{2 * tp + fp + fn} \)
\end{center}



\chapter{\IfLanguageName{dutch}{Model Resultaten}{Model Results}}%
\label{ch:model-resultaten}

In dit hoofdstuk worden de model resultaten weergeven.

\section{\IfLanguageName{dutch}{Evaluatie criteria}{Evaluation criteria}}%
\label{sec:evaluatie-criteria}

In deze tabel zijn de individuele resultaten van elke model evaluatie weergeven.

\begin{center}
  \begin{tabular}{ | l | l | l | l | l | l |}
    \hline
    Model & Accuracy & Precision & Recall & IoU & F1 \\ \hline
    CGNET & 0.7736 & 0.5432 & 0.1547 & 0.1369 & 0.2408 \\ \hline
    ChangeFormer & 0.7628 & 0.4534 & 0.1077 & 0.0953 & 0.1740 \\ \hline
    Random Forest & 0.8515 & 0.2909 & 0.3018 & 0.1684 & 0.2844 \\ \hline
    SVM & placeholder & placeholder & placeholder & placeholder & placeholder \\ \hline
  \end{tabular}
\end{center}

\section{\IfLanguageName{dutch}{Visualisatie}{Visualisation}}%
\label{sec:evaluatie-criteria}

Hier komt in een volgende versie van de BP een visualisatie overzicht.

% Voeg hier je eigen hoofdstukken toe die de ``corpus'' van je bachelorproef
% vormen. De structuur en titels hangen af van je eigen onderzoek. Je kan bv.
% elke fase in je onderzoek in een apart hoofdstuk bespreken.

%\input{...}
%\input{...}
%...

\input{conclusie}

%---------- Bijlagen -----------------------------------------------------------

\appendix

\chapter{Onderzoeksvoorstel}

Het onderwerp van deze bachelorproef is gebaseerd op een onderzoeksvoorstel dat vooraf werd beoordeeld door de promotor. Dat voorstel is opgenomen in deze bijlage.

%% TODO: 
%\section*{Samenvatting}

% Kopieer en plak hier de samenvatting (abstract) van je onderzoeksvoorstel.
Geospatiale remote sensing biedt innovatieve methoden voor het monitoren van veranderingen in het landschap. 
Dit onderzoek richt zich specifiek op de detectie en visualisatie van veranderingen in vegetatiebedekking in de stad Gent tussen 2020 en 2024.
Hiervoor worden Sentinel-2 satellietbeelden gebruikt en geanalyseerd met behulp van machine learning-technieken. 
Een geavanceerde machine learning-pijplijn wordt ontwikkeld om deze veranderingen te detecteren en te interpreteren.
De onderzoeksmethode omvat een vergelijkende studie van diverse algoritmen, waaronder Support Vector Machines (SVM), Random Forests, 
Convolutional Neural Networks (CNN), Siamese Networks en Vision Transformers. Preprocessing van de data, 
zoals herschalen en semantische segmentatie, zorgt ervoor dat de inputdata consistent en geschikt is voor training. 
De algoritmen worden beoordeeld op nauwkeurigheid, consistentie en vermogen om visueel interpreteerbare resultaten te genereren.
Een belangrijke evaluatie metriek is de Vegetation Condition Index (VCI), die veranderingen in vegetatiebedekking kwantificeert door 
Normalized Difference Vegetation Index (NDVI)-waarden te vergelijken. Daarnaast worden metrieken zoals precision, recall en 
Intersection over Union (IoU) toegepast om de prestaties van de modellen te beoordelen. Het onderzoek beoogt niet alleen inzicht te bieden 
in de geschiktheid van verschillende machine learning-modellen voor geospatiale change detection, maar ook bij te dragen aan de 
toepassing van AI in stedelijke milieuanalyse en besluitvorming. Het onderzoek fungeert ook als oproep tot actie voor natuurbehoud in steden.

% Verwijzing naar het bestand met de inhoud van het onderzoeksvoorstel
%---------- Inleiding ---------------------------------------------------------

\section{Introductie}%
\label{sec:introductie}

%Waarover zal je bachelorproef gaan? Introduceer het thema en zorg dat volgende zaken zeker duidelijk aanwezig zijn:

%\begin{itemize}
%  \item kaderen thema
%  \item de doelgroep
%  \item de probleemstelling en (centrale) onderzoeksvraag
%  \item de onderzoeksdoelstelling
%\end{itemize}

%Denk er aan: een typische bachelorproef is \textit{toegepast onderzoek}, wat betekent dat je start vanuit een concrete probleemsituatie in bedrijfscontext, een \textbf{casus}. Het is belangrijk om je onderwerp goed af te bakenen: je gaat voor die \textit{ene specifieke probleemsituatie} op zoek naar een goede oplossing, op basis van de huidige kennis in het vakgebied.

%De doelgroep moet ook concreet en duidelijk zijn, dus geen algemene of vaag gedefinieerde groepen zoals \emph{bedrijven}, \emph{developers}, \emph{Vlamingen}, enz. Je richt je in elk geval op it-professionals, een bachelorproef is geen populariserende tekst. Eén specifiek bedrijf (die te maken hebben met een concrete probleemsituatie) is dus beter dan \emph{bedrijven} in het algemeen.

%Formuleer duidelijk de onderzoeksvraag! De begeleiders lezen nog steeds te veel voorstellen waarin we geen onderzoeksvraag terugvinden.

%Schrijf ook iets over de doelstelling. Wat zie je als het concrete eindresultaat van je onderzoek, naast de uitgeschreven scriptie? Is het een proof-of-concept, een rapport met aanbevelingen, \ldots Met welk eindresultaat kan je je bachelorproef als een succes beschouwen?
\textcolor{hogent-purple}{Geo-ICT is de toepassing van informatie- en communicatietechnologie in de geografie. Een belangrijk component van Geo-ICT is het programma GIS.
Geografische informatiesystemen (GIS) zijn digitale systemen die geografische gegevens analyseren, visualiseren en beheren om ruimtelijke 
patronen en relaties te begrijpen. Het wordt vandaag de dag door bedrijven gebruikt in verschillende sectoren. 
Nu er vraag is naar GIS-applicaties, zijn er ook ondernemingen die zich specialiseren in deze trend, zoals bijvoorbeeld het bedrijf GeoAI. 
Zij combineren Geo-ICT met het experimentele veld van machine learning. Dit kan ingezet worden om verschillende doeleinden te bereiken, zoals
bijvoorbeeld automatische kaartgeneratie, voorspellende modellering, \ldots.}
\newline
Het thema waar ik me op focus is geospatiale remote sensing change detection. Dit gaat over het ontdekken van veranderingen in een bepaald landschap.
Hier komt de sterkte van AI naar boven. Het is in staat patronen te ontdekken die wij als mensen moeilijk begrijpen.
Deze studie is gericht op het bestuderen van de aanwezigheid van vegetatie in de stad Gent tussen de periode van 2020-2024. 
De dataset bestaat uit afbeeldingen genomen door de Sentinel 2 satelliet. Tijdens het onderzoek ga ik verschillende machine learning 
algoritmes (Support Vector Machines (SVM), Random Forests, Convolutional Neural Networks (CNNs), Siamese Networks en Vision transformer) 
testen op de dataset. Het model moet in staat zijn vegetatie te detecteren en ook het verschil tussen de twee periodes te visualiseren. 
Het open-sourceplatform QGIS wordt gebruikt om de data te verkrijgen en te visualiseren.
\newline
\textcolor{hogent-yellow}{De onderzoeksvraag: Welk machine learning-model is het meest geschikt voor het nauwkeurig detecteren en visualiseren van de
veranderingen in aanwezigheid van vegetatie in de stad Gent tussen 2020 en 2024, met behulp van remote sensing change detection?}
\newline
\textcolor{hogent-darkgreen}{De deelvragen:
\newline
- Welke object klasse (vegetatie, bebouwing, wegen enz \ldots) toont de grootste veranderingen in percentage tussen de twee tijdstippen?
\newline
- Zijn er specifieke patronen in de veranderingen, zoals groei, afname of stabiliteit van groen in bepaalde delen van de stad?
\newline
- Zijn er ruimtelijke patronen in de veranderingen van vegetatie in de stad Gent?
\newline
- Is er een verschil tussen handmatige feature-extractie en automatische extractie op basis van deep learning, en wat is de impact hiervan op de resultaten?
\newline
- Welke evaluatie methoden zijn het meest geschikt om de prestaties van het model te beoordelen?
\newline
- Hoe wordt de balans gemeten tussen nauwkeurigheid en andere prestatie-indicatoren, zoals recall of precision?
\newline
- Welke criteria bepalen of een model als `acceptabel' wordt beschouwd?
\newline
- Is één dataset voldoende om de generaliseerbaarheid van het model te garanderen?
\newline
- Hoe wordt omgegaan met mogelijke bias of beperkingen in de gebruikte datasets?
\newline
- Welke preprocessing-stappen zijn nodig om de datasets geschikt te maken voor model input?}



%---------- Stand van zaken ---------------------------------------------------

\section{State-of-the-art}%
\label{sec:state-of-the-art}

% Hier beschrijf je de \emph{state-of-the-art} rondom je gekozen onderzoeksdomein, d.w.z.\ een inleidende, doorlopende tekst over het onderzoeksdomein van je bachelorproef. Je steunt daarbij heel sterk op de professionele \emph{vakliteratuur}, en niet zozeer op populariserende teksten voor een breed publiek. Wat is de huidige stand van zaken in dit domein, en wat zijn nog eventuele open vragen (die misschien de aanleiding waren tot je onderzoeksvraag!)?

% Je mag de titel van deze sectie ook aanpassen (literatuurstudie, stand van zaken, enz.). Zijn er al gelijkaardige onderzoeken gevoerd? Wat concluderen ze? Wat is het verschil met jouw onderzoek?

% Verwijs bij elke introductie van een term of bewering over het domein naar de vakliteratuur, bijvoorbeeld~\autocite{Hykes2013}! Denk zeker goed na welke werken je refereert en waarom.

% Draag zorg voor correcte literatuurverwijzingen! Een bronvermelding hoort thuis \emph{binnen} de zin waar je je op die bron baseert, dus niet er buiten! Maak meteen een verwijzing als je gebruik maakt van een bron. Doe dit dus \emph{niet} aan het einde van een lange paragraaf. Baseer nooit teveel aansluitende tekst op eenzelfde bron.

% Als je informatie over bronnen verzamelt in JabRef, zorg er dan voor dat alle nodige info aanwezig is om de bron terug te vinden (zoals uitvoerig besproken in de lessen Research Methods).

% Change detecion
Change detection (CD) is het waarnemen van verandering in een bepaald gebied aan de hand van afbeeldingen genomen op verschillende tijden 
\autocite{SINGH_1989}. Machine learning wordt al geruime tijd toegepast op CD. Maar in de laatste jaren heeft de opkomst van verschillende 
nieuwere algoritmen het veld verbreed. Zo worden verschillende studies gedaan rond de impact van deep learning neurale netwerken op CD,
zoals het onderzoek van \textcite{Bai_2022} over Deep Learning Change Detection (DLCD). Deep learning brengt een groot voordeel met zich 
mee, namelijk dat het in staat is om automatische feature extractie toe te passen op ruwe data zoals bewezen in de studie van \textcite{LeCun_2015}.
De meest voorkomende algoritmen zijn de CNN, Vision transformer en Siamese neurale netwerken. 
% Siamese networks
Siamese networks zijn voor het eerst voorgesteld in het onderzoek van \textcite{NIPS1993_288cc0ff}. Het is een algoritme dat bestaat uit
twee of meerdere identieke neurale netwerken met dezelfde weights en biases die samenkomen tot één output \autocite{Serrano_2023}. 
Een van de grootste voordelen ligt in het feit dat het in staat is accurate predicties te maken met weinig input data. 
Het wordt daarom ook wel het one-shot model genoemd \autocite{koch2015siamese}. 
% CNN
Convolutional neurale netwerken worden gezien als één van de beste machine learning algoritmes als het aankomt op visualisatietaken. De afbeelding wordt opgedeeld 
in kleine deeltjes aan de hand van filters, die dan allemaal geanalyseerd worden op een kleinere schaal \autocite{Geron2022}. Dit maakt het ideaal om hierarchische 
features en patronen te ontdekken in de input data.
% Vision Transformer
Transformers zijn neurale netwerken die uitsluitend bestaan uit attention layers voorgesteld in het onderzoek van \textcite{Vaswani2017}.
Attention is een methode om alleen relevante input data te verwerken in plaats van de hele batch \autocite{Geron2022}. 
Dit vermindert de trainingstijd zonder de nauwkeurigheid van het model significant te beïnvloeden. Vision transformer (ViT) is 
een transformer gebouwd voor visuele taken. Oorspronkelijk geïntroduceerd in de studie van \textcite{dosovitskiy2020image}. 
Vision Transformers (ViTs) zijn, dankzij de transformer-architectuur, bijzonder goed afgestemd op grootschalige datasets, 
wat bijdraagt aan hun huidige populariteit.
% Non Deep Learning modellen
Buiten deep learning zijn er ook nog andere AI modellen die gebruikt worden voor CD.
Deze studie focust op twee modellen: Support Vector Machines (SVM) en Random Forests.
% SVM
SVM's, oorspronkelijk geïntroduceerd in het onderzoek van \textcite{cortes1995support}, maken gebruik van feature extraction om zo een 
decision boundary (hyperlane) te bouwen die de veranderingen in het gebied scheidt van het onveranderde deel. 
Er zijn al verschillende studies gedaan rondom het gebruik van support vector machines voor CD, zoals \textcite{bovolo2008novel} en \textcite{Habib_2009}
% Random forests
Random forests is een ensemble learning model dat bestaat uit meerdere decision trees, 
waarbij elke tree gebaseerd is op willekeurige vectoren met dezelfde verdeling \autocite{Breiman_2001}. Meerdere studies zoals 
\textcite{Wessels_2016} en \textcite{Feng_2018} tonen aan dat random Forest geschikt is voor change detection.


% Voor literatuurverwijzingen zijn er twee belangrijke commando's:
% \autocite{KEY} => (Auteur, jaartal) Gebruik dit als de naam van de auteur
%   geen onderdeel is van de zin.
% \textcite{KEY} => Auteur (jaartal)  Gebruik dit als de auteursnaam wel een
%   functie heeft in de zin (bv. ``Uit onderzoek door Doll & Hill (1954) bleek
%   ...'')

%---------- Methodologie ------------------------------------------------------
\section{Methodologie}%
\label{sec:methodologie}

% Hier beschrijf je hoe je van plan bent het onderzoek te voeren. Welke onderzoekstechniek ga je toepassen om elk van je onderzoeksvragen te beantwoorden? Gebruik je hiervoor literatuurstudie, interviews met belanghebbenden (bv.~voor requirements-analyse), experimenten, simulaties, vergelijkende studie, risico-analyse, PoC, \ldots?

% Valt je onderwerp onder één van de typische soorten bachelorproeven die besproken zijn in de lessen Research Methods (bv.\ vergelijkende studie of risico-analyse)? Zorg er dan ook voor dat we duidelijk de verschillende stappen terug vinden die we verwachten in dit soort onderzoek!

% Vermijd onderzoekstechnieken die geen objectieve, meetbare resultaten kunnen opleveren. Enquêtes, bijvoorbeeld, zijn voor een bachelorproef informatica meestal \textbf{niet geschikt}. De antwoorden zijn eerder meningen dan feiten en in de praktijk blijkt het ook bijzonder moeilijk om voldoende respondenten te vinden. Studenten die een enquête willen voeren, hebben meestal ook geen goede definitie van de populatie, waardoor ook niet kan aangetoond worden dat eventuele resultaten representatief zijn.

% Uit dit onderdeel moet duidelijk naar voor komen dat je bachelorproef ook technisch voldoen\-de diepgang zal bevatten. Het zou niet kloppen als een bachelorproef informatica ook door bv.\ een student marketing zou kunnen uitgevoerd worden.

% Je beschrijft ook al welke tools (hardware, software, diensten, \ldots) je denkt hiervoor te gebruiken of te ontwikkelen.

% Probeer ook een tijdschatting te maken. Hoe lang zal je met elke fase van je onderzoek bezig zijn en wat zijn de concrete \emph{deliverables} in elke fase?


Het uiteindelijke doel van dit onderzoek is het ontwikkelen van een machine learning-pijplijn die effectief veranderingen 
kan detecteren tussen de aanwezigheid van vegetatie tussen twee tijdstippen. Dit is een vergelijkende studie, 
waarbij verschillende machine learning modellen getest worden. 
De eerste taak is het samenstellen van twee datasets opgebouwd uit satellietfoto's van Sentinel 2. 
Dataset 1 zal bestaan uit afbeeldingen van Gent in 2020 (t₁).  
De tweede dataset bestaat uit satellietfoto's van Gent in 2024 (t₂). 
Hiermee gaan we veranderingen proberen waar te nemen door ze te vergelijken met de afbeeldingen van t₁ met behulp van machine learning.
De datasets zijn opgedeeld in drie delen: trainging (60\%), validatie (20\%) en testing (20\%).
Training om het model te trainen. Validatie om te controleren hoe goed het model werkt op data die het niet eerder heeft gezien. 
Dit zorgt ervoor dat het model beter generialiseerd tijdens de testing fase. Testing om het model te evalueren. 
Bovendien moet de dataset goed gerepresenteerd zijn, zodat de accuraatheid van het model niet lijdt onder gelijkaardige situaties 
(het model moet in staat zijn de objecten te detecteren tijdens de verschillende seizoenen).
Elk vooraf gedefinieerd model wordt getraind en getest op de data. Het is niet de bedoeling om ze vanaf nul op te bouwen. Er wordt 
gebruik gemaakt van pretrained modellen. Dit deel van de studie zal herhaaldelijk worden toegepast op de dataset. Eerst passen we 
preprocessing toe op de datasets. In dit geval wordt herschalen toegepast om alle afbeeldingen dezelfde grootte te geven (Data Augmentatie). 
De data wordt gelabeld aan de hand van semantische segmentatie. Een computer vision taak waarbij elke pixel tot een bepaalde klasse/object wordt 
geclassifiseerd. Dit resulteerd in een segmentatie kaart waarin elk object (bv. vegetatie, gebouw, straat enz.) wordt aangeduid. Objecten 
van t₁ worden vergeleken met objecten van t₂. De veranderingen worden gedetecteerd door gemiddelde kleur, grootte en 
verschil in vorm van een object. Vervolgens worden de verschillen tussen t₁ en t₂ geclassifiseerd. Hierbij wordt niet alleen 
verandering aangeduid maar ook waarin objecten zijn veranderd. Voor de evaluatie worden een aantal metrieken gebruikt. De belangrijkste 
hieronder is de vegetation condition index (VCI). De VCI vergelijkt de NDVI-waarden van twee tijdstippen (t₁ en t₂) om veranderingen 
in vegetatiebedekking te berekenen. Andere evaluatie methoden zoals confusiematrix-methoden (precision en recall) en IoU (Intersection over Union)
worden ook gebruikt. De twee belangrijkste python machine learning frameworks die gebruikt zullen worden tijdens deze studie zijn TensorFlow en scikit-learn.


%---------- Verwachte resultaten ----------------------------------------------
\section{Verwacht resultaat, conclusie}%
\label{sec:verwachte_resultaten}

% Hier beschrijf je welke resultaten je verwacht. Als je metingen en simulaties uitvoert, kan je hier al mock-ups maken van de grafieken samen met de verwachte conclusies. Benoem zeker al je assen en de onderdelen van de grafiek die je gaat gebruiken. Dit zorgt ervoor dat je concreet weet welk soort data je moet verzamelen en hoe je die moet meten.

% Wat heeft de doelgroep van je onderzoek aan het resultaat? Op welke manier zorgt jouw bachelorproef voor een meerwaarde?

% Hier beschrijf je wat je verwacht uit je onderzoek, met de motivatie waarom. Het is \textbf{niet} erg indien uit je onderzoek andere resultaten en conclusies vloeien dan dat je hier beschrijft: het is dan juist interessant om te onderzoeken waarom jouw hypothesen niet overeenkomen met de resultaten.

Voor het technische aspect van deze studie wordt verwacht dat de deep learning modellen, zoals Convolutional Neural Networks (CNN), 
Vision Transformers en Siamese netwerken, betere prestaties zullen leveren in vergelijking met Support Vector Machines (SVM) en 
Random Forests. Recente onderzoeken ondersteunen deze verwachting door aan te tonen dat deep learning algoritmen doorgaans 
superieure resultaten behalen, met name bij complexe taken zoals geospatiale veranderingendetectie.
\newline
Met betrekking tot de veranderingen in vegetatiebedekking binnen Gent wordt een lichte toename verwacht in vergelijking met 2020, 
gebaseerd op trends in stedelijke vergroening en gerelateerde ecologische initiatieven in de regio.
\newline
Dit onderzoek gaat over het detecteren van veranderingen in de vegetatie in een stedelijk gebied. Het is gericht naar organisaties 
die het behoud van de natuur op zich nemen. Zo kunnen rapporten met zulke studies een call to action worden voor het verder 
behouden van onze natuur.

%%---------- Andere bijlagen --------------------------------------------------
% TODO: Voeg hier eventuele andere bijlagen toe. Bv. als je deze BP voor de
% tweede keer indient, een overzicht van de verbeteringen t.o.v. het origineel.
%\input{...}

%%---------- Backmatter, referentielijst ---------------------------------------

\backmatter{}

\setlength\bibitemsep{2pt} %% Add Some space between the bibliograpy entries
\printbibliography[heading=bibintoc]

\end{document}
