%%=============================================================================
%% Samenvatting
%%=============================================================================

% TODO: De "abstract" of samenvatting is een kernachtige (~ 1 blz. voor een
% thesis) synthese van het document.
%
% Een goede abstract biedt een kernachtig antwoord op volgende vragen:
%
% 1. Waarover gaat de bachelorproef?
% 2. Waarom heb je er over geschreven?
% 3. Hoe heb je het onderzoek uitgevoerd?
% 4. Wat waren de resultaten? Wat blijkt uit je onderzoek?
% 5. Wat betekenen je resultaten? Wat is de relevantie voor het werkveld?
%
% Daarom bestaat een abstract uit volgende componenten:
%
% - inleiding + kaderen thema
% - probleemstelling
% - (centrale) onderzoeksvraag
% - onderzoeksdoelstelling
% - methodologie
% - resultaten (beperk tot de belangrijkste, relevant voor de onderzoeksvraag)
% - conclusies, aanbevelingen, beperkingen
%
% LET OP! Een samenvatting is GEEN voorwoord!

%%---------- Nederlandse samenvatting -----------------------------------------
%
% TODO: Als je je bachelorproef in het Engels schrijft, moet je eerst een
% Nederlandse samenvatting invoegen. Haal daarvoor onderstaande code uit
% commentaar.
% Wie zijn bachelorproef in het Nederlands schrijft, kan dit negeren, de inhoud
% wordt niet in het document ingevoegd.

% \IfLanguageName{english}{%
% \selectlanguage{dutch}
% \chapter*{Samenvatting}
% \lipsum[1-4]
% \selectlanguage{english}
% }{}

%%---------- Samenvatting -----------------------------------------------------
% De samenvatting in de hoofdtaal van het document

\chapter*{\IfLanguageName{dutch}{Samenvatting}{Abstract}}

Het abstract wordt uitgewerkt in een latere versie van de BP

% Geospatiale remote sensing biedt innovatieve methoden voor het monitoren van veranderingen in het landschap. 
% Dit onderzoek richt zich specifiek op de detectie en visualisatie van veranderingen in vegetatiebedekking in de stad Gent tussen 2020 en 2024.
% Hiervoor worden Sentinel-2 satellietbeelden gebruikt en geanalyseerd met behulp van machine learning-technieken. 
% Een geavanceerde machine learning-pijplijn wordt ontwikkeld om deze veranderingen te detecteren en te interpreteren.
% De onderzoeksmethode omvat een vergelijkende studie van diverse algoritmen, waaronder Support Vector Machines (SVM), Random Forests, 
% Convolutional Neural Networks (CNN), Siamese Networks en Vision Transformers. Preprocessing van de data, 
% zoals herschalen en semantische segmentatie, zorgt ervoor dat de inputdata consistent en geschikt is voor training. 
% De algoritmen worden beoordeeld op nauwkeurigheid, consistentie en vermogen om visueel interpreteerbare resultaten te genereren.
% Een belangrijke evaluatie metriek is de Vegetation Condition Index (VCI), die veranderingen in vegetatiebedekking kwantificeert door 
% Normalized Difference Vegetation Index (NDVI)-waarden te vergelijken. Daarnaast worden metrieken zoals precision, recall en 
% Intersection over Union (IoU) toegepast om de prestaties van de modellen te beoordelen. Het onderzoek beoogt niet alleen inzicht te bieden 
% in de geschiktheid van verschillende machine learning-modellen voor geospatiale change detection, maar ook bij te dragen aan de 
% toepassing van AI in stedelijke milieuanalyse en besluitvorming. Het onderzoek fungeert ook als oproep tot actie voor natuurbehoud in steden.
