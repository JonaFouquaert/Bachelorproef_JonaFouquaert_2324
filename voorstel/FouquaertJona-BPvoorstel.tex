%==============================================================================
% Sjabloon onderzoeksvoorstel bachproef
%==============================================================================
% Gebaseerd op document class `hogent-article'
% zie <https://github.com/HoGentTIN/latex-hogent-article>

% Voor een voorstel in het Engels: voeg de documentclass-optie [english] toe.
% Let op: kan enkel na toestemming van de bachelorproefcoördinator!
\documentclass{hogent-article}

% Invoegen bibliografiebestand
\addbibresource{voorstel.bib}

% Informatie over de opleiding, het vak en soort opdracht
\studyprogramme{Professionele bachelor toegepaste informatica}
\course{Bachelorproef}	
\assignmenttype{Onderzoeksvoorstel}
% Voor een voorstel in het Engels, haal de volgende 3 regels uit commentaar
% \studyprogramme{Bachelor of applied information technology}
% \course{Bachelor thesis}
% \assignmenttype{Research proposal}

\academicyear{2024-2025} % TODO: pas het academiejaar aan

% TODO: Werktitel
%\title{AI zal relaties tussen verschillenden objecten binnen een geografische/ruimtelijke omgeving vaststellen en situaties herkennen.}
\title{Diepgaande Analyse van Geospatiale Veranderingen: Een Vergelijkende Studie van Machine Learning Modellen in Remote Sensing Change Detection voor de Vegetatie van Gent (2020-2024)}

% TODO: Studentnaam en emailadres invullen
\author{Jona Fouquaert}
\email{jona.fouquaert@student.hogent.be}

% TODO: Medestudent
% Gaat het om een bachelorproef in samenwerking met een student in een andere
% opleiding? Geef dan de naam en emailadres hier
% \author{Yasmine Alaoui (naam opleiding)}
% \email{yasmine.alaoui@student.hogent.be}

% TODO: Geef de co-promotor op
\supervisor[Co-promotor]{C. Stal (Hogent, \href{mailto:cornelis.stal@hogent.be}{cornelis.stal@hogent.be})}

% Binnen welke specialisatierichting uit 3TI situeert dit onderzoek zich?
% Kies uit deze lijst:
%
% - Mobile \& Enterprise development
% - AI \& Data Engineering
% - Functional \& Business Analysis
% - System \& Network Administrator
% - Mainframe Expert
% - Als het onderzoek niet past binnen een van deze domeinen specifieer je deze zelf

\specialisation{AI \& Data Engineering}
\keywords{Change detection (CD), GEO-ICT, QGIS, Machine learning, Natuurbehoud}

\begin{document}

\begin{abstract}

% Dit onderzoek bestudeert geavanceerde machine learning-modellen voor het detecteren van vegetatieveranderingen in Gent (2020-2024). 
% De focus ligt op geospatiale remote sensing change detection in een stedelijke omgeving, waarbij diverse algoritmes, waaronder SVM, Random Forests, CNNs, 
% Siamese Networks en Vision Transformers worden getest op Sentinel 2-beelden. QGIS wordt gebruikt voor gegevensverwerking en visualisatie.
% De onderzoeksvraag richt zich op veranderingen in vegetatie in Gent tussen 2020 en 2024, met de nadruk op het meest geschikte
% machine learning-model voor nauwkeurige detectie en visualisatie. Deelvragen behandelen gebieden met grote veranderingen en 
% ruimtelijke patronen. De state-of-the-art sectie benadrukt de evolutie van change detection, met aandacht voor deep learning 
% (CNNs, Vision Transformers en Siamese networks) en traditionele modellen zoals SVM en Random Forests.
% De methodologie omvat de bouw van twee datasets uit Sentinel 2-beelden, semi-automatische labeling met QGIS, en het gebruik van 
% pre-trained modellen met iteratieve preprocessing, training en testing. Evaluatie gebeurt op nauwkeurigheid, ondersteund door 
% hyperparameter tuning. Verwacht wordt dat deep learning-modellen superieure resultaten zullen opleveren, met potentieel belang voor 
% natuurbehoudsorganisaties. Het onderzoek fungeert als oproep tot actie voor natuurbescherming.

Geospatiale remote sensing biedt innovatieve methoden voor het monitoren van veranderingen in het landschap. 
Dit onderzoek richt zich specifiek op de detectie en visualisatie van veranderingen in vegetatiebedekking in de stad Gent tussen 2020 en 2024.
Hiervoor worden Sentinel-2 satellietbeelden gebruikt en geanalyseerd met behulp van machine learning-technieken. 
Een geavanceerde machine learning-pijplijn wordt ontwikkeld om deze veranderingen te detecteren en te interpreteren.
De onderzoeksmethode omvat een vergelijkende studie van diverse algoritmen, waaronder Support Vector Machines (SVM), Random Forests, 
Convolutional Neural Networks (CNN), Siamese Networks en Vision Transformers. Preprocessing van de data, 
zoals herschalen en semantische segmentatie, zorgt ervoor dat de inputdata consistent en geschikt is voor training. 
De algoritmen worden beoordeeld op nauwkeurigheid, consistentie en vermogen om visueel interpreteerbare resultaten te genereren.
Een belangrijke evaluatie metriek is de Vegetation Condition Index (VCI), die veranderingen in vegetatiebedekking kwantificeert door 
Normalized Difference Vegetation Index (NDVI)-waarden te vergelijken. Daarnaast worden metrieken zoals precision, recall en 
Intersection over Union (IoU) toegepast om de prestaties van de modellen te beoordelen. Het onderzoek beoogt niet alleen inzicht te bieden 
in de geschiktheid van verschillende machine learning-modellen voor geospatiale change detection, maar ook bij te dragen aan de 
toepassing van AI in stedelijke milieuanalyse en besluitvorming. Het onderzoek fungeert ook als oproep tot actie voor natuurbehoud in steden.

% Dit onderzoek combineert innovatieve technologieën, 
% zoals deep learning en remote sensing, met een focus op praktische toepassingen in een stedelijke context, en biedt daarmee een waardevolle 
% bijdrage aan de verdere ontwikkeling van Geo-ICT en machine learning.

\end{abstract}

\tableofcontents

% De hoofdtekst van het voorstel zit in een apart bestand, zodat het makkelijk
% kan opgenomen worden in de bijlagen van de bachelorproef zelf.
%---------- Inleiding ---------------------------------------------------------

\section{Introductie}%
\label{sec:introductie}

%Waarover zal je bachelorproef gaan? Introduceer het thema en zorg dat volgende zaken zeker duidelijk aanwezig zijn:

%\begin{itemize}
%  \item kaderen thema
%  \item de doelgroep
%  \item de probleemstelling en (centrale) onderzoeksvraag
%  \item de onderzoeksdoelstelling
%\end{itemize}

%Denk er aan: een typische bachelorproef is \textit{toegepast onderzoek}, wat betekent dat je start vanuit een concrete probleemsituatie in bedrijfscontext, een \textbf{casus}. Het is belangrijk om je onderwerp goed af te bakenen: je gaat voor die \textit{ene specifieke probleemsituatie} op zoek naar een goede oplossing, op basis van de huidige kennis in het vakgebied.

%De doelgroep moet ook concreet en duidelijk zijn, dus geen algemene of vaag gedefinieerde groepen zoals \emph{bedrijven}, \emph{developers}, \emph{Vlamingen}, enz. Je richt je in elk geval op it-professionals, een bachelorproef is geen populariserende tekst. Eén specifiek bedrijf (die te maken hebben met een concrete probleemsituatie) is dus beter dan \emph{bedrijven} in het algemeen.

%Formuleer duidelijk de onderzoeksvraag! De begeleiders lezen nog steeds te veel voorstellen waarin we geen onderzoeksvraag terugvinden.

%Schrijf ook iets over de doelstelling. Wat zie je als het concrete eindresultaat van je onderzoek, naast de uitgeschreven scriptie? Is het een proof-of-concept, een rapport met aanbevelingen, \ldots Met welk eindresultaat kan je je bachelorproef als een succes beschouwen?
\textcolor{hogent-purple}{Geo-ICT is de toepassing van informatie- en communicatietechnologie in de geografie. Een belangrijk component van Geo-ICT is het programma GIS.
Geografische informatiesystemen (GIS) zijn digitale systemen die geografische gegevens analyseren, visualiseren en beheren om ruimtelijke 
patronen en relaties te begrijpen. Het wordt vandaag de dag door bedrijven gebruikt in verschillende sectoren. 
Nu er vraag is naar GIS-applicaties, zijn er ook ondernemingen die zich specialiseren in deze trend, zoals bijvoorbeeld het bedrijf GeoAI. 
Zij combineren Geo-ICT met het experimentele veld van machine learning. Dit kan ingezet worden om verschillende doeleinden te bereiken, zoals
bijvoorbeeld automatische kaartgeneratie, voorspellende modellering, \ldots.}
\newline
Het thema waar ik me op focus is geospatiale remote sensing change detection. Dit gaat over het ontdekken van veranderingen in een bepaald landschap.
Hier komt de sterkte van AI naar boven. Het is in staat patronen te ontdekken die wij als mensen moeilijk begrijpen.
Deze studie is gericht op het bestuderen van de aanwezigheid van vegetatie in de stad Gent tussen de periode van 2020-2024. 
De dataset bestaat uit afbeeldingen genomen door de Sentinel 2 satelliet. Tijdens het onderzoek ga ik verschillende machine learning 
algoritmes (Support Vector Machines (SVM), Random Forests, Convolutional Neural Networks (CNNs), Siamese Networks en Vision transformer) 
testen op de dataset. Het model moet in staat zijn vegetatie te detecteren en ook het verschil tussen de twee periodes te visualiseren. 
Het open-sourceplatform QGIS wordt gebruikt om de data te verkrijgen en te visualiseren.
\newline
\textcolor{hogent-yellow}{De onderzoeksvraag: Welk machine learning-model is het meest geschikt voor het nauwkeurig detecteren en visualiseren van de
veranderingen in aanwezigheid van vegetatie in de stad Gent tussen 2020 en 2024, met behulp van remote sensing change detection?}
\newline
\textcolor{hogent-darkgreen}{De deelvragen:
\newline
- Welke object klasse (vegetatie, bebouwing, wegen enz \ldots) toont de grootste veranderingen in percentage tussen de twee tijdstippen?
\newline
- Zijn er specifieke patronen in de veranderingen, zoals groei, afname of stabiliteit van groen in bepaalde delen van de stad?
\newline
- Zijn er ruimtelijke patronen in de veranderingen van vegetatie in de stad Gent?
\newline
- Is er een verschil tussen handmatige feature-extractie en automatische extractie op basis van deep learning, en wat is de impact hiervan op de resultaten?
\newline
- Welke evaluatie methoden zijn het meest geschikt om de prestaties van het model te beoordelen?
\newline
- Hoe wordt de balans gemeten tussen nauwkeurigheid en andere prestatie-indicatoren, zoals recall of precision?
\newline
- Welke criteria bepalen of een model als `acceptabel' wordt beschouwd?
\newline
- Is één dataset voldoende om de generaliseerbaarheid van het model te garanderen?
\newline
- Hoe wordt omgegaan met mogelijke bias of beperkingen in de gebruikte datasets?
\newline
- Welke preprocessing-stappen zijn nodig om de datasets geschikt te maken voor model input?}



%---------- Stand van zaken ---------------------------------------------------

\section{State-of-the-art}%
\label{sec:state-of-the-art}

% Hier beschrijf je de \emph{state-of-the-art} rondom je gekozen onderzoeksdomein, d.w.z.\ een inleidende, doorlopende tekst over het onderzoeksdomein van je bachelorproef. Je steunt daarbij heel sterk op de professionele \emph{vakliteratuur}, en niet zozeer op populariserende teksten voor een breed publiek. Wat is de huidige stand van zaken in dit domein, en wat zijn nog eventuele open vragen (die misschien de aanleiding waren tot je onderzoeksvraag!)?

% Je mag de titel van deze sectie ook aanpassen (literatuurstudie, stand van zaken, enz.). Zijn er al gelijkaardige onderzoeken gevoerd? Wat concluderen ze? Wat is het verschil met jouw onderzoek?

% Verwijs bij elke introductie van een term of bewering over het domein naar de vakliteratuur, bijvoorbeeld~\autocite{Hykes2013}! Denk zeker goed na welke werken je refereert en waarom.

% Draag zorg voor correcte literatuurverwijzingen! Een bronvermelding hoort thuis \emph{binnen} de zin waar je je op die bron baseert, dus niet er buiten! Maak meteen een verwijzing als je gebruik maakt van een bron. Doe dit dus \emph{niet} aan het einde van een lange paragraaf. Baseer nooit teveel aansluitende tekst op eenzelfde bron.

% Als je informatie over bronnen verzamelt in JabRef, zorg er dan voor dat alle nodige info aanwezig is om de bron terug te vinden (zoals uitvoerig besproken in de lessen Research Methods).

% Change detecion
Change detection (CD) is het waarnemen van verandering in een bepaald gebied aan de hand van afbeeldingen genomen op verschillende tijden 
\autocite{SINGH_1989}. Machine learning wordt al geruime tijd toegepast op CD. Maar in de laatste jaren heeft de opkomst van verschillende 
nieuwere algoritmen het veld verbreed. Zo worden verschillende studies gedaan rond de impact van deep learning neurale netwerken op CD,
zoals het onderzoek van \textcite{Bai_2022} over Deep Learning Change Detection (DLCD). Deep learning brengt een groot voordeel met zich 
mee, namelijk dat het in staat is om automatische feature extractie toe te passen op ruwe data zoals bewezen in de studie van \textcite{LeCun_2015}.
De meest voorkomende algoritmen zijn de CNN, Vision transformer en Siamese neurale netwerken. 
% Siamese networks
Siamese networks zijn voor het eerst voorgesteld in het onderzoek van \textcite{NIPS1993_288cc0ff}. Het is een algoritme dat bestaat uit
twee of meerdere identieke neurale netwerken met dezelfde weights en biases die samenkomen tot één output \autocite{Serrano_2023}. 
Een van de grootste voordelen ligt in het feit dat het in staat is accurate predicties te maken met weinig input data. 
Het wordt daarom ook wel het one-shot model genoemd \autocite{koch2015siamese}. 
% CNN
Convolutional neurale netwerken worden gezien als één van de beste machine learning algoritmes als het aankomt op visualisatietaken. De afbeelding wordt opgedeeld 
in kleine deeltjes aan de hand van filters, die dan allemaal geanalyseerd worden op een kleinere schaal \autocite{Geron2022}. Dit maakt het ideaal om hierarchische 
features en patronen te ontdekken in de input data.
% Vision Transformer
Transformers zijn neurale netwerken die uitsluitend bestaan uit attention layers voorgesteld in het onderzoek van \textcite{Vaswani2017}.
Attention is een methode om alleen relevante input data te verwerken in plaats van de hele batch \autocite{Geron2022}. 
Dit vermindert de trainingstijd zonder de nauwkeurigheid van het model significant te beïnvloeden. Vision transformer (ViT) is 
een transformer gebouwd voor visuele taken. Oorspronkelijk geïntroduceerd in de studie van \textcite{dosovitskiy2020image}. 
Vision Transformers (ViTs) zijn, dankzij de transformer-architectuur, bijzonder goed afgestemd op grootschalige datasets, 
wat bijdraagt aan hun huidige populariteit.
% Non Deep Learning modellen
Buiten deep learning zijn er ook nog andere AI modellen die gebruikt worden voor CD.
Deze studie focust op twee modellen: Support Vector Machines (SVM) en Random Forests.
% SVM
SVM's, oorspronkelijk geïntroduceerd in het onderzoek van \textcite{cortes1995support}, maken gebruik van feature extraction om zo een 
decision boundary (hyperlane) te bouwen die de veranderingen in het gebied scheidt van het onveranderde deel. 
Er zijn al verschillende studies gedaan rondom het gebruik van support vector machines voor CD, zoals \textcite{bovolo2008novel} en \textcite{Habib_2009}
% Random forests
Random forests is een ensemble learning model dat bestaat uit meerdere decision trees, 
waarbij elke tree gebaseerd is op willekeurige vectoren met dezelfde verdeling \autocite{Breiman_2001}. Meerdere studies zoals 
\textcite{Wessels_2016} en \textcite{Feng_2018} tonen aan dat random Forest geschikt is voor change detection.


% Voor literatuurverwijzingen zijn er twee belangrijke commando's:
% \autocite{KEY} => (Auteur, jaartal) Gebruik dit als de naam van de auteur
%   geen onderdeel is van de zin.
% \textcite{KEY} => Auteur (jaartal)  Gebruik dit als de auteursnaam wel een
%   functie heeft in de zin (bv. ``Uit onderzoek door Doll & Hill (1954) bleek
%   ...'')

%---------- Methodologie ------------------------------------------------------
\section{Methodologie}%
\label{sec:methodologie}

% Hier beschrijf je hoe je van plan bent het onderzoek te voeren. Welke onderzoekstechniek ga je toepassen om elk van je onderzoeksvragen te beantwoorden? Gebruik je hiervoor literatuurstudie, interviews met belanghebbenden (bv.~voor requirements-analyse), experimenten, simulaties, vergelijkende studie, risico-analyse, PoC, \ldots?

% Valt je onderwerp onder één van de typische soorten bachelorproeven die besproken zijn in de lessen Research Methods (bv.\ vergelijkende studie of risico-analyse)? Zorg er dan ook voor dat we duidelijk de verschillende stappen terug vinden die we verwachten in dit soort onderzoek!

% Vermijd onderzoekstechnieken die geen objectieve, meetbare resultaten kunnen opleveren. Enquêtes, bijvoorbeeld, zijn voor een bachelorproef informatica meestal \textbf{niet geschikt}. De antwoorden zijn eerder meningen dan feiten en in de praktijk blijkt het ook bijzonder moeilijk om voldoende respondenten te vinden. Studenten die een enquête willen voeren, hebben meestal ook geen goede definitie van de populatie, waardoor ook niet kan aangetoond worden dat eventuele resultaten representatief zijn.

% Uit dit onderdeel moet duidelijk naar voor komen dat je bachelorproef ook technisch voldoen\-de diepgang zal bevatten. Het zou niet kloppen als een bachelorproef informatica ook door bv.\ een student marketing zou kunnen uitgevoerd worden.

% Je beschrijft ook al welke tools (hardware, software, diensten, \ldots) je denkt hiervoor te gebruiken of te ontwikkelen.

% Probeer ook een tijdschatting te maken. Hoe lang zal je met elke fase van je onderzoek bezig zijn en wat zijn de concrete \emph{deliverables} in elke fase?


Het uiteindelijke doel van dit onderzoek is het ontwikkelen van een machine learning-pijplijn die effectief veranderingen 
kan detecteren tussen de aanwezigheid van vegetatie tussen twee tijdstippen. Dit is een vergelijkende studie, 
waarbij verschillende machine learning modellen getest worden. 
De eerste taak is het samenstellen van twee datasets opgebouwd uit satellietfoto's van Sentinel 2. 
Dataset 1 zal bestaan uit afbeeldingen van Gent in 2020 (t₁).  
De tweede dataset bestaat uit satellietfoto's van Gent in 2024 (t₂). 
Hiermee gaan we veranderingen proberen waar te nemen door ze te vergelijken met de afbeeldingen van t₁ met behulp van machine learning.
De datasets zijn opgedeeld in drie delen: trainging (60\%), validatie (20\%) en testing (20\%).
Training om het model te trainen. Validatie om te controleren hoe goed het model werkt op data die het niet eerder heeft gezien. 
Dit zorgt ervoor dat het model beter generialiseerd tijdens de testing fase. Testing om het model te evalueren. 
Bovendien moet de dataset goed gerepresenteerd zijn, zodat de accuraatheid van het model niet lijdt onder gelijkaardige situaties 
(het model moet in staat zijn de objecten te detecteren tijdens de verschillende seizoenen).
Elk vooraf gedefinieerd model wordt getraind en getest op de data. Het is niet de bedoeling om ze vanaf nul op te bouwen. Er wordt 
gebruik gemaakt van pretrained modellen. Dit deel van de studie zal herhaaldelijk worden toegepast op de dataset. Eerst passen we 
preprocessing toe op de datasets. In dit geval wordt herschalen toegepast om alle afbeeldingen dezelfde grootte te geven (Data Augmentatie). 
De data wordt gelabeld aan de hand van semantische segmentatie. Een computer vision taak waarbij elke pixel tot een bepaalde klasse/object wordt 
geclassifiseerd. Dit resulteerd in een segmentatie kaart waarin elk object (bv. vegetatie, gebouw, straat enz.) wordt aangeduid. Objecten 
van t₁ worden vergeleken met objecten van t₂. De veranderingen worden gedetecteerd door gemiddelde kleur, grootte en 
verschil in vorm van een object. Vervolgens worden de verschillen tussen t₁ en t₂ geclassifiseerd. Hierbij wordt niet alleen 
verandering aangeduid maar ook waarin objecten zijn veranderd. Voor de evaluatie worden een aantal metrieken gebruikt. De belangrijkste 
hieronder is de vegetation condition index (VCI). De VCI vergelijkt de NDVI-waarden van twee tijdstippen (t₁ en t₂) om veranderingen 
in vegetatiebedekking te berekenen. Andere evaluatie methoden zoals confusiematrix-methoden (precision en recall) en IoU (Intersection over Union)
worden ook gebruikt. De twee belangrijkste python machine learning frameworks die gebruikt zullen worden tijdens deze studie zijn TensorFlow en scikit-learn.


%---------- Verwachte resultaten ----------------------------------------------
\section{Verwacht resultaat, conclusie}%
\label{sec:verwachte_resultaten}

% Hier beschrijf je welke resultaten je verwacht. Als je metingen en simulaties uitvoert, kan je hier al mock-ups maken van de grafieken samen met de verwachte conclusies. Benoem zeker al je assen en de onderdelen van de grafiek die je gaat gebruiken. Dit zorgt ervoor dat je concreet weet welk soort data je moet verzamelen en hoe je die moet meten.

% Wat heeft de doelgroep van je onderzoek aan het resultaat? Op welke manier zorgt jouw bachelorproef voor een meerwaarde?

% Hier beschrijf je wat je verwacht uit je onderzoek, met de motivatie waarom. Het is \textbf{niet} erg indien uit je onderzoek andere resultaten en conclusies vloeien dan dat je hier beschrijft: het is dan juist interessant om te onderzoeken waarom jouw hypothesen niet overeenkomen met de resultaten.

Voor het technische aspect van deze studie wordt verwacht dat de deep learning modellen, zoals Convolutional Neural Networks (CNN), 
Vision Transformers en Siamese netwerken, betere prestaties zullen leveren in vergelijking met Support Vector Machines (SVM) en 
Random Forests. Recente onderzoeken ondersteunen deze verwachting door aan te tonen dat deep learning algoritmen doorgaans 
superieure resultaten behalen, met name bij complexe taken zoals geospatiale veranderingendetectie.
\newline
Met betrekking tot de veranderingen in vegetatiebedekking binnen Gent wordt een lichte toename verwacht in vergelijking met 2020, 
gebaseerd op trends in stedelijke vergroening en gerelateerde ecologische initiatieven in de regio.
\newline
Dit onderzoek gaat over het detecteren van veranderingen in de vegetatie in een stedelijk gebied. Het is gericht naar organisaties 
die het behoud van de natuur op zich nemen. Zo kunnen rapporten met zulke studies een call to action worden voor het verder 
behouden van onze natuur.

\printbibliography[heading=bibintoc]

\end{document}